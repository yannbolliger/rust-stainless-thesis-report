\documentclass[10pt, aspectratio=149]{beamer}

\usetheme[progressbar=frametitle, block=fill]{metropolis}

%\setbeamertemplate{frame footer}{Yann Bolliger -- \textit{Frequent Subgraph Search}}

\usepackage{appendixnumberbeamer}

\usepackage{booktabs}

\usepackage{pgfplots}
\usepackage{svg}
\usepgfplotslibrary{dateplot}

\usepackage{tikz}
\usetikzlibrary{graphs, graphdrawing,graphdrawing.layered, graphdrawing.force, quotes}

\usepackage{xspace}
\newcommand{\themename}{\textbf{\textsc{metropolis}}\xspace}

\title{Frequent Subgraph Search}
\subtitle{On deterministic and probablistic graphs}
\date{25.06.20}
\author{Yann Bolliger}
\institute{
COM-512 Networks out of control \\
Prof. Matthias Grossglauser and Prof. Patrick Thiran \\
École polytechnique fédérale de Lausanne
}
\titlegraphic{\includesvg[height=0.5cm]{epfl-logo.svg}}

\begin{document}

\maketitle

\begin{frame}{Contents}
  \setbeamertemplate{section in toc}[sections numbered]
  
  \begin{columns}[T]
  \column{0.5\textwidth}
  \tableofcontents%[hideallsubsections]
  
  
  \column{0.5\textwidth}
  \textbf{Based on}
  \begin{itemize}
      \item  Kuramochi et al., \textit{An efficient algorithm for discovering frequent subgraphs} \cite{FSG} 
      
      \item Yuan et al., \textit{Efficient Subgraph Similarity Search on Large Probabilistic Graph Databases} \cite{sim}
  \end{itemize}
  
  
  \end{columns}
\end{frame}

\section[Intro]{Introduction}

\begin{frame}{Labelled Graphs}
  \begin{itemize}
    
    \item Deterministic, labelled, undirected graph:
    $G = (V, E, \Sigma, L)$ with labelling $L: V \cup E \rightarrow \Sigma$, \cite{sim}.
   \end{itemize}
   
   \only<-2>{
   \begin{columns}[T]
    \column{0.3\textwidth}
   $(V, E)=$
  \begin{figure}
     \begin{tikzpicture}
     \graph[nodes={draw,circle}]{
        a__ -- b__ -- c__;
        b -- d__ -- a;
     };
     \end{tikzpicture}
   \end{figure}
   
    \column{0.3\textwidth}<2>
    $\Sigma = \{a,b,c,d, x,y\}$
   \end{columns}
   }
   
   \only<3>{
    \begin{figure}
     \begin{tikzpicture}
     \graph[nodes={draw,circle}]{
        a -- b --["x"] c;
        b -- d --["y"] a;
     };
     \end{tikzpicture}
   \end{figure}
   }
\end{frame}

\begin{frame}{Application}
\begin{itemize}
    \item Database of graphs (transactions):
    $\mathcal{D} = \{G_1, G_2, \ldots \}$
\end{itemize}

\only<1>{
  \begin{figure}
  \centering
  \begin{tikzpicture}
    \graph[layered layout, nodes={draw,circle}]{
    a -- { b, c, d }; 
    b -- { c, e };
    c -- { f };
    i__f -- { l__a, j__e }; 
    j -- { m__b }; 
    l -- {o__c, m} ;
    m -- o ; 
  };
  \end{tikzpicture}
\end{figure}
}
\only<2>{
  \begin{figure}
    \centering
    \includegraphics[width=0.55\textwidth]{img/silico.png}
    \caption{Protein-protein interaction network \cite{silico}.}
  \end{figure}
} \only<3>{
  \begin{figure}
    \centering
    \includegraphics[width=0.6\textwidth]{img/schizo1.jpg}
    \caption{The schizophrenia interactome, genes as nodes and PPIs as edges  \cite{schizo}.}
  \end{figure}
}
\end{frame}

\begin{frame}{Find Subgraphs}

We want to find \textbf{common patterns in the database}.

 \begin{columns}[T]
\column{0.5\textwidth}
\centering
\begin{figure}
     \begin{tikzpicture}
     \graph[spring layout, nodes={draw,circle}]{
        a -- b -- c;
        b -- d -- a;
        g__a -- h__b;
     };
     \end{tikzpicture}
   \end{figure}
   Yes.

\column{0.5\textwidth}
\centering
\begin{figure}
     \begin{tikzpicture}
     \graph[spring layout, nodes={draw,circle}]{
        a -- b -- c;
        b -- d -- a;
        g__a -- h__c;
     };
     \end{tikzpicture}
   \end{figure}
   No.
\end{columns}
\end{frame}

\begin{frame}{More precisely}
  \begin{block}{Subgraph Isomorphism \cite{sim}}
    Given two deterministic, labelled, undirected graphs $G_1, G_2$, we say $G_1$ is subgraph isomorphic to $G_2$,
    $$
    G_1 \subseteq_{iso} G_2
    $$ $\Leftrightarrow \ \exists$ an injection $f: V_1 \rightarrow V_2$ such that: 
     \begin{itemize}
         \item $G_1$ is an unlabelled subgraph of $G_2$
         
         \item the vertex labels are preserved
         
         \item  the edge labels are preserved
     \end{itemize}
  \end{block}
\end{frame}


\section[Discovering frequent subgraphs]{Frequent Subgraph Search}

\subsection{The Problem}

\begin{frame}{For deterministic graphs}

\begin{alertblock}{Frequent Subgraph Discovery Problem as in \cite{FSG}}
Given 

\begin{itemize}
    \item database $\mathcal{D} = \{G_1, G_2, \ldots \}$
    \item \textit{support threshold} $\sigma \in (0, 1]$,
\end{itemize}
 find all connected graphs that occur as subgraphs in at least $\sigma \cdot |\mathcal{D}|$ input graphs.
\end{alertblock}

\begin{quote}
    Return all common subgraph patterns (in most of the database).
\end{quote}

A naive solution is to solve the subgraph isomorphism problem (which is in NP-complete \cite{npcomp}) many times for each $G \in \mathcal{D}$.
    
\end{frame}

\subsection{High-level Algorithm}

\begin{frame}{The FSG Algorithm}
The high-level idea \cite{FSG}:

\begin{enumerate}
    \item Start with all frequent single-edge subgraphs.
    \item Smartly generate size-$(k+1)$ candidate subgraphs from the previous frequent size-$k$ subgraphs.
    \item Smartly count the frequency and prune infrequent ones.
    \item Go back to 2.) unless no frequent subgraph was found in this iteration.
\end{enumerate}
\end{frame}

\begin{frame}{Step-by-step}
Toy example with $\mathcal{D} = \{G_1, G_2\}$ and $\sigma = 1$.

\begin{columns}[T]
\only<2->{
\column{0.5\textwidth}

\begin{itemize}
    \only<2-3> {
    \item Common subgraphs with one edge: \\
    \begin{figure}
    \begin{tikzpicture}[inner sep=0.05cm, level distance=0.05cm, sibling distance=0.05cm]
    \graph[layered layout, nodes={draw,circle}]{
    a -- b;
    z__a -- c;
    y__b -- w__c; 
    x__b -- e;
    };
    \end{tikzpicture}
    \end{figure}
    }
    
    \only<3-4>{
     \item 
        \only<3>{Join only 1-subgraphs with a common node:} 
        \only<4>{1-subgraphs:} 
    \begin{figure}
    \begin{tikzpicture}[inner sep=0.025cm, level distance=0.025cm, sibling distance=0.025cm]
    \graph[layered layout, nodes={draw,circle}]{
    a -- {b, c};
    z__b -- {y__c, x__a};
    w__c -- {q__a, r__b}; 
    };
    \end{tikzpicture} \\
    \begin{tikzpicture}[inner sep=0.025cm, level distance=0.025cm, sibling distance=0.025cm]
    \graph[layered layout, nodes={draw,circle}]{
    s__b -- {e, t__c};
    u__b -- {v__e, k__a};
    };
    \end{tikzpicture}
    \end{figure}
    }
    
    \only<4-5>{
    \item  
        \only<4>{Join only 2-subgraphs with a common 1-subgraph:} 
        \only<5>{2-subgraphs:} 
    \begin{figure}
    \begin{tikzpicture}[inner sep=0.025cm, level distance=0.025cm, sibling distance=0.025cm]
    \graph[layered layout, nodes={draw,circle}]{
    a -- {b, c};
    b -- e;
    
    x__a -- y__b -- z__c -- x;
    
    h__b -- {i__a, j__c, k__e};
    };
    \end{tikzpicture}
    \end{figure}
    }
    
    \only<5>{
    \item Join only 3-subgraphs with a common 2-subgraph: 
    \begin{figure}
    \begin{tikzpicture}[inner sep=0.025cm]
    \graph[spring layout, horizontal=e to b, node distance=0.8cm, nodes={draw,circle}]{
        e -- b;
        b -- {c, a};
        a -- c;
    };
    \end{tikzpicture}
    \end{figure}
    }
\end{itemize}

}

\column{0.5\textwidth}
\begin{figure}
  \centering
  \begin{tikzpicture}
    \graph[layered layout, nodes={draw,circle}]{
    a -- { b, c, d }; 
    b -- { c, e };
    c -- { f };
    i__f -- { l__a, j__e }; 
    j -- { m__b }; 
    l -- {o__c, m} ;
    m -- o ; 
  };
  \end{tikzpicture}
  \caption{$\mathcal{D} = \{G_1, G_2\}$}
\end{figure}

\end{columns}
\end{frame}


\subsection{Specialities}

\begin{frame}{FSG's strengths}

The algorithm contains two main achievements  that make it efficient by
taking full advantage of the \textbf{downward closed property} \cite{FSG}:

\begin{enumerate}
    \item \textbf{Iterative candidate generation} \\ 
    Only create a size-$(k + 1)$ candidates by joining frequent size-$k$ candidates, that have a \textit{common size-$(k-1)$ subgraph}.
    
    \item \textbf{Canonical labelling} \\ 
    Uniquely labels subgraphs such that isomorphic candidates have the same label, to avoid regenerating the same candidate multiple times.
\end{enumerate}

Plus some more \textit{downward closed} property related optimisations for the frequency counting.
\end{frame}




\section[Subgraph Similarity Search on Probabilistic Graphs]
{Probabilistic subgraph similarity}

\subsection{The Problem}

\begin{frame}{A query algorithm}
We want to find out which graphs have a certain substructure.

\begin{figure}
    \centering
    \includegraphics[width=0.5\textwidth]{img/query.png}
    \caption{This algorithm takes a query as input. From \cite{structural}.}
\end{figure}
    
\end{frame}

\begin{frame}{For Probabilistic Graphs}
    The second paper \cite{sim} was the first to treat this problem on graphs such that
    
    \begin{columns}[T]
    \column{0.5\textwidth}
    \begin{itemize}
        \item a probabilistic graph $g = (g^c, X_E)$ for $g^c = (V, E, \Sigma, L)$ the corresponding deterministic graph.
        
        \item<2-> the edge probabilities $X_E$ are \textit{not} independent but correlate in small neighbour edge sets,
        
        \item<3-> those follow a joint density $\mathbb{P}(x_{ne})$\\
        for $x_{ne} = (x_1, x_2, \ldots x_k)$ of $X_E$.
    \end{itemize}
    
    \column{0.5\textwidth}
    \begin{figure}
    \centering
    \begin{tikzpicture}[inner sep=0.025cm]
    \graph[spring layout, horizontal=e to b, node distance=1cm, nodes={draw,circle}]{
        e --["i"] b;
        b --["j"] c;
        b --["k"] a;
        a --["l"] c;
    };
    \end{tikzpicture}
    \caption{Underlying $g^c$.}
    \end{figure}
    
    \only<3->{
    \begin{itemize}
        \item Triangle joint density $\forall (x_j, x_k, x_l) \in X_E: \mathbb{P}(x_j, x_k, x_l)$.
        
        \item $b$'s neighbours joint density $\forall (x_i,x_j,x_k) \in X_E: \mathbb{P}(x_i,x_j,x_k)$.
    \end{itemize}
    }
    
    \end{columns}
\end{frame}

\begin{frame}{Only similar}
   We relax the subgraph isomorphism with a \textit{subgraph distance threshold} $\delta$.

\vspace{0.3cm}
 \begin{columns}[T]
    \centering
  \column{0.2\textwidth}
  Graph $q$: \\ 
  \begin{figure} 
  \centering
  \begin{tikzpicture}[inner sep=0.025cm, level distance=0.5cm]
    \graph[layered layout, nodes={draw,circle}]{
        b -- {c, a};
    };
  \end{tikzpicture}
  \end{figure}
  
  \column{0.3\textwidth}
  $U = \{q_1, \ldots, q_a\}$ the set of $q$'s subgraphs if removed $\delta = 1$ edge:  \\ 
   \begin{figure} 
  \centering
  \begin{tikzpicture}[inner sep=0.025cm]
    \graph[spring layout, node distance=0.5cm, nodes={draw,circle}]{
         b -- c;
        z__b -- a;
    };
  \end{tikzpicture}
  \end{figure}
  
  \column{0.2\textwidth}
  $\exists q_s \in U$ s.t. $q_s \subseteq_{iso} p$:
   \begin{figure} 
  \centering
  \begin{tikzpicture}[inner sep=0.025cm, level distance=0.5cm]
    \graph[layered layout, node distance=0.5cm, nodes={draw,circle}]{
        a -- {b, c};
    };
  \end{tikzpicture}
  \end{figure}
  \end{columns}
  
  \vspace{0.3cm}
  \begin{block}{Subgraph similarity}<2>
  Two deterministic graphs are \textbf{subgraph similar} $\mathbf{q \subseteq_{sim} p}$ if any ($|q| - \delta$)-subgraph of $q$ is subgraph isomorphic to $p$.
  \end{block}
\end{frame}


\begin{frame}{And probable}
  \begin{itemize}
    \item<1-> A sample from a probabilistic graph is a \textit{possible world graph}, denoted $g \Rightarrow g'$, $g' \in PWG(g)$.

    \item<2-> $SIM(q, g) = \{g' |g' \in PWG(g), q \subseteq_{sim} g' \}$ is the set of subgraph similar possible world graphs of $g$.
  \end{itemize}
    
  \begin{block}<3->{Subgraph Similarity Probability (SSP)}
  For a query graph $g$ and a probabilistic graph $g$, the SSP is
    $$
    \mathbb{P}(q \subseteq_{sim} g) = \sum_{g' \in SIM(q, g)} \mathbb{P}(g \Rightarrow g')
    $$
  \end{block}
\end{frame}


\begin{frame}{The Problem}

\begin{alertblock}{Threshold-based probabilistic subgraph similarity matching (T-PS)}
Given 
\begin{itemize}
    \item a database of probabilistic graphs $\mathcal{D} = \{g_1, g_2, \ldots\}$
    
    \item a deterministic \textbf{query graph} $q$
    
    \item a \textit{subgraph distance threshold} $\delta \in \mathbb{N}$
    
    \item a \textit{probability threshold} $\epsilon \in (0,1]$
\end{itemize}

return $A_q = \{ g | g \in \mathcal{D}, \mathbb{P}(q \subseteq_{sim} g) \geq \epsilon \}$
\end{alertblock}

\begin{quote}
    Return all graphs that contain something similar to $q$.
\end{quote}

This problem is \#P-complete \cite{sim} and the algorithm returns an approximate answer.

\end{frame}

\subsection{High-level Algorithm}

\begin{frame}{The T-PS Algorithm}

In each step, the set of graphs from $\mathcal{D}$ is reduced until $A_q$ is remaining \cite{sim}.

\begin{enumerate}
    \item Structural pruning \\ 
    Remove all graphs don't contain $q$ for sure. Deterministic algorithm from another paper \cite{structural}.
    
    \item \textbf{Probabilistic pruning}  \\
    Using the PMI \textit{Probabilistic Matrix Index}.
    
    \item Verification \\
    Verify results with Monte Carlo sampling.
\end{enumerate}
\end{frame}


\subsection{Specialities}

\begin{frame}{Probabilistic Matrix Index}
An extension of the technique used in \cite{structural}.

  \begin{itemize}
      \item Create \textbf{feature set} $F$ of small graphs using the algorithm from \cite{structural}.
      
      \item Calculate upper and lower bounds on $\mathbb{P}(f \subseteq_{sim} g), \ \forall (f,g) \in F \times \mathcal{D}$.
  \end{itemize}
  
  \vspace{0.3cm}
  
  \begin{columns}
    \column{0.5\textwidth}
    \begin{figure}
    \centering
    \begin{tikzpicture}[inner sep=0.025cm]
    \graph[spring layout, nodes={draw,circle}]{
      a__ --["b"] b__;
      
      i__ --["a"] k__ --["b"] l__;
      
      x__ --["a"] y__ --["c"] z__;
      y --["b"]  w__;
    };
    \end{tikzpicture}
    \caption{Features $f_1$, $f_2, f_3$}
    \end{figure}

    \column{0.5\textwidth}
    \centering
    \only<1>{
    \begin{figure}
    \centering
    \begin{tikzpicture}[inner sep=0.025cm]
    \graph[spring layout, nodes={draw,circle}]{
      a__ --["a"] b__ --["b"] d__ --["d"] a__;
      
      h__ --["a"] i__;
      j__ --["b"] i --["a"] k__ --["b"] l__;
      k -- ["c"] m__;
    };
    \end{tikzpicture}
    \caption{$g_1$, $g_2$}
    \end{figure}
    }
    
    \only<2>{
      \begin{tabular}{|l|lll|}
  \hline
  {} & $g_1$ & $g_2$ & $\cdots$ \\
  \hline
  $f_1$  & (0.55, 0.64) & (0.42, 0.5)  & {} \\
  $f_2$  & (0.3, 0.48) & (0.26, 0.58)  & {} \\
  $f_3$  & (-) & (0.08, 0.15)  & {} \\
  $\vdots$ & {} & {} & $\ddots$ \\
  \hline
  \end{tabular}
}
  \end{columns}

Example taken from \cite{sim}.
\end{frame}


\begin{frame}{Pruning conditions}
Let $U = \{q_1, \ldots, q_a\}$ the set of graphs after $q$ is relaxed by $\delta$ edges.

\begin{block}{Probabilistic pruning}
  \begin{enumerate}
      \item For each $q_i$ take features $f_i^1, \  f_i^2 \in F$ such that $f_i^1, f_i^2 \subseteq_{iso} g^c$ and
      $$
      f_i^1 \subseteq_{iso} q_i \subseteq_{iso} f_i^2.
      $$
      
      \item If $\sum_{i=1}^a UpperB(f_i^1) < \epsilon$ then $g$ can be pruned.
      
      \item If $\sum_{i=1}^a LowerB(f_i^2) -  \sum_{1\leq i,j \leq a} UpperB(f_i^2)UpperB(f_j^2) \geq \epsilon$ then $g \in A_q$.
  \end{enumerate}
\end{block}

 The algorithm minimises the gaps of the lower and upper bounds by transforming the feature choice to two well known problems: \textit{weighted set cover} and \textit{integer quadratic programming}.
\end{frame}



\section{Conclusion}

\begin{frame}{Similarities}
 Both algorithms 
    \begin{itemize}
        \item solve some form of subgraph isomorphism problem.
        
        \item need to minimise the number of isomorphism calculations.
        
        \item therefore use the \textit{filter-and-verify} approach.
    \end{itemize}
\end{frame}

\begin{frame}{Differences}
    \begin{columns}[T,onlytextwidth]
    \column{0.5\textwidth}
    
    \textbf{FSG}
    \begin{itemize}
        \item Deterministic graphs in $\mathcal{D}$
        
        \item Only isomorphic, i.e. $\delta = 0$
        
        \item \textbf{Mining algorithm}: Find and return \textit{subgraphs} that are common, no query graph.
    \end{itemize}
    
    \column{0.5\textwidth}
    
    \textbf{T-PS}
    \begin{itemize}
        \item Probabilistic graphs in $\mathcal{D}$
        
        \item $\delta > 0$
        
        \item \textbf{Query system}: Return all $g \in \mathcal{D}$ that are probably subgraph similar to the input query graph $q$.
    \end{itemize}
    
    \end{columns}
    
    \vspace{6pt}
    Could use FSG to discover common patterns in the deterministic graphs $g^c, g \in \mathcal{D}$, then see where the patterns occur with T-PS.
\end{frame}


{\setbeamercolor{palette secondary}{fg=black, bg=yellow}
\begin{frame}[standout]
  Questions?
\end{frame}
}

{\setbeamercolor{palette secondary}{fg=black, bg=yellow}
\begin{frame}[standout]
  Thank you!
\end{frame}
}


\appendix

\begin{frame}{FSG Candidate Generation}
    \begin{itemize}
        \item For a size-$k$ subgraph $F_i$, let $\mathcal{P}(F_i) = \{ H_{i,1}, H_{i,2} \}$ the set of size-$(k-1)$ subgraphs with the smallest and second smallest canonical label – the \textit{primary subgraphs} of $F_i$.
        
        \item FSG only joins $F_i, F_j$ if $\mathcal{P}(F_i) \cap \mathcal{P}(F_j) \neq \emptyset$. This still generates all the candidates.
    \end{itemize}

    \begin{block}{Theorem 1 in \cite{FSG}}
      Given a connected size-$(k+1)$ valid candidate subgraph $C$, there exists a pair of connected size-$k$ frequent subgraphs $F_i, F_j$ such that $\mathcal{P}(F_i) \cap \mathcal{P}(F_j) \neq \emptyset$, that can be joined with respect to their common primary subgraph to obtain $C$.
    \end{block}
\end{frame}

\begin{frame}{FSG Canonical Labelling}
\begin{figure}
    \centering
    \includegraphics[width=0.9\textwidth]{img/labelling.png}
    \caption{A sample graph and three possible labels. From \cite{FSG}.}
\end{figure}
\end{frame}


\begin{frame}{Probabilistic Pruning theorems}

    \begin{block}{Lemma 1 in \cite{sim}}
      $U = \{q_1, \ldots, q_a\}$ the set of graphs after $q$ is relaxed by $\delta$ edges. $Bq_i$ is an Boolean variable that is true if $q_i \subseteq_{iso} g^c$.
      
      $$
      \mathbb{P}(q \subseteq_{sim} g) = \mathbb{P}(Bq_1 \vee \ldots \vee Bq_a)
      $$
    \end{block}
    
       \begin{block}{Theorem 3 in \cite{sim}}
      Given a probability threshold $\epsilon$, if $\sum_{i=1}^a UpperB(f_i^1) < \epsilon$ then $g$ can be safely pruned.
    \end{block}
    
       \begin{block}{Theorem 4 in \cite{sim}}
      Given a probability threshold $\epsilon$, if $\sum_{i=1}^a LowerB(f_i^2) -  \sum_{1\leq i,j \leq a} UpperB(f_i^2)UpperB(f_j^2) \geq \epsilon$ then $g \in A_q$.
    \end{block}
\end{frame}

\begin{frame}{Probabilistic Pruning theorems}
     \begin{block}{Theorem 3 in \cite{sim}}
      Given a probability threshold $\epsilon$, if $\sum_{i=1}^a UpperB(f_i^1) < \epsilon$ then $g$ can be safely pruned.
    \end{block}
    
    \textbf{Proof} \\
    Since $f_i \subseteq_{iso} q_i \Rightarrow Bq_i \subseteq Bf_i$
    \begin{align*}
      \mathbb{P}(q \subseteq_{sim} g) 
      &= \mathbb{P}(Bq_1 \vee \ldots \vee Bq_a)
      \\
      &\leq \mathbb{P}(Bf_1^1 \vee \ldots \vee Bf_a^1)
      \\
     &\leq \mathbb{P}(Bf_1^1) + \ldots + \mathbb{P}(Bf_a^1)
     \\
     &\leq UpperB(Bf_1^1) + \ldots + UpperB(Bf_a^1) < \epsilon.
    \end{align*}
\end{frame}

\begin{frame}[allowframebreaks]{References}

  \bibliography{bib}
  \bibliographystyle{abbrv}

\end{frame}

\end{document}

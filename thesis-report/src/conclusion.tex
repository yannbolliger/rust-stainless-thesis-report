With this thesis, I presented Rust-Stainless, a formal verification tool for
Rust based on the Stainless verifier. Building on a solid foundation of existing
software and infrastructure, I added many features to the tool and thus
increased its usability and expressiveness significantly. Along with smaller
language features, the most important additions are immutable references and
boxes, the Rust trait to Scala type class translation and, first and foremost,
the mutability translation. Further, this project includes the theoretical
reasoning behind the correctness of the mutability translation as well as
numerous bugfixes and additions to both Rust-Stainless and Scala Stainless.

The development went hand-in-hand with testing and evaluating the tool on more
and more complex Rust examples. As an intern at Informal Systems, I had access
to experienced Rust developers who advised me on the choice of features to
support. Chapter \ref{evaluation} shows how the tool was evaluated against some
code of Informal Systems and it also demonstrates that the translations
performed by our tool are very fast, in comparison to the time it takes to
verify the code.

By combining Rust's borrow checking with Stainless's verification,
Rust-Stainless gains an unimagined expressiveness. Thanks to Rust features like
mutability and \lstinline!mem::replace! along with Stainless's new
\lstinline!freshCopy! primitive, the tool can now process code that would fail
Stainless's aliasing restrictions in Scala but is proven memory safe by the Rust
compiler. The support for in-place updates with mutability unlocks the
possibility of verifying efficient Rust code with Stainless, like a search tree
-- without using duplicate functional ghost structures for verification.


\paragraph{Extraction}

There are still some indispensable language features that Rust-Stainless needs
to support before it can extract idiomatic Rust in all its facets. The primary
ones are loops, arrays, and closures. However, the biggest limitation of the
tool  is the one-crate-limitation inherited from the compile model of Rust.
Solving it would overcome many related problems. Instead of supporting vectors,
one could add sensible contracts for vectors and iterators. More generally, with
the ability to attach contracts to crate-external items, Rust-Stainless could
finally  use standard library items like traits for equality, ordering and
hashing. Additionally, it could provide real implementations in the Stainless
crate, instead of just providing bindings and replacing them with generated
implementations in the extraction. This would also enable the use of verified
data structures like the list-map.

\paragraph{Mutability}

The mutability translation presented in this thesis can theoretically translate
all Rust mutability to Scala, but the current implementation is limited by the
imperative phase of Stainless. With the current backend, this project has
probably maximised the features that can be supported. Even if the imperative
phase was bug-free, there are still the two drastic limitations of recursive
functions returning mutable references and mutable references in mutable
variables.

The promising solution to that problem could be the new fully imperative phase
that is currently developed for Stainless. That phase does not have aliasing
restrictions but rather its own notion of heap references. Hence, one could
remove some of the Stainless specific changes to the translation, making it
simpler. Additionally, I implemented the mutable cell synthesis in such a way
that it would be easy to target the new heap reference cells instead.


\paragraph{Counter-Examples}

Even if Rust-Stainless seems to be the only formal verification project for Rust
that translates from the THIR, instead of the lower-level MIR or LLVM-IR, this
thesis project showed the decision to be productive. Most Rust constructs can be
translated in a one-to-one fashion to Stainless AST. Even more so,
Rust-Stainless could turn  that decision into a virtue by becoming the first
Rust verification tool that is counter-example complete! As Stainless is
counter-example complete, it would suffice to translate the Scala
counter-examples back to Rust.


\paragraph{New Applications}

The ability of verifying high-level properties on Rust code sparks numerous
ideas for new applications of Rust-Stainless. For example, if software needs to
be verified and very efficient, one can write it in Scala, verify it with
Stainless and then use Stainless's C-code-generation feature. The generated code
is memory safe but not as optimised as hand-written C. A much simpler approach
would be to use Rust-Stainless  to directly verify the efficient Rust code,
given that it supports all the used language features.

Another idea is to use Rust-Stainless to verify implementations that are
compiled to \emph{Web Assembly (WASM)}. For example, the IBC relayer of Informal
Systems reads policies that govern its functioning from WASM. Rust-Stainless
could  be used to guarantee certain properties on these policies, e.g. that no
rules are in conflict with each other.


\hfill \break \noindent  In summary, with this thesis and the implementation of
the tool, Rust-Stainless, I showed that the combination of Rust and Stainless is
very powerful. While the tool has its limitations, it also offers many promising
paths for future work and  it can already be used to verify real-world Rust
code, making systems more correct and safer.

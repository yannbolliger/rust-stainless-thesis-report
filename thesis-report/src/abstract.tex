Writing correct software is hard, yet in systems that have a high failure cost
or are not easily upgraded, like blockchains, bugs and security problems cannot
be tolerated. Therefore, these systems are prime use cases for \emph{formal
verification}, the task of mathematically proving that a system conforms to its
specification. Many recent blockchains are implemented in the Rust programming
language because it guarantees memory safety at compile-time while providing
full control over efficiency to the programmer.

To explore the powerful combination of the safety guarantees of Rust's compiler
with the ability to formally verify high-level properties on programs, we
present \emph{Rust-Stainless}. Stainless is a formal verification tool for Scala
built on an SMT solver backend. Rust-Stainless is a frontend for Stainless that
extracts a subset of Rust, translates it to Scala and verifies it with
Stainless. In this thesis project, I significantly advance the feature set of
Rust-Stainless, increasing its expressiveness to programs that cannot be
processed in Scala Stainless. In particular, this thesis adds support for
mutability and in-place updates, by introducing a theoretical translation from
Rust to Scala. I argue that the translation yields equivalent runtime semantics
in both languages and refine it to work with Stainless. This allows to verify
efficiently implemented data structures in Rust. Other new features are traits
with verified contracts, references, and heap allocation.

The tool is evaluated on real-world Rust examples from a blockchain context.
Remaining limitations of the approach concern missing support for some Rust
language features, the limitation of only processing one crate at a time, and
certain restrictions of the current Stainless backend. Finally, I explore
possible promising ways for the future advancement of Rust-Stainless.

\section{Background}

\subsection{Rust compiler}

Our tool translates Rust code to Stainless. To make this manageable and minimise
the amount of work the translation needs to do, we heavily rely on the Rust
compiler \cite{rustc-guide}, \passthrough{\lstinline!rustc!}. Normally, the
compiler would take Rust source code as input and produce an executable binary.
This process involves multiple phases in each of which the high-level input
source is transformed or \emph{lowered} into a more low-level representation.
We'll continue by giving a brief overview of the compile process.

The Rust compiler always operates at the level of a single crate. That
is, each crate is compiled separately and maybe linked to other crates
only after compilation. This means that the \emph{intermediate
representations} like the HIR described afterwards are only available
for items of the current crate. The exception is for some metadata of
public items like function signatures that need to be accessible across
crates.

The first two phases of the compile process are concerned with
transforming the input source into a token stream -- lexing -- and then
into an \emph{abstract syntax tree (AST)} -- parsing. First syntactical
validations, name resolutions and macro expansion are performed on the
AST. While the AST is already a representation of the program, it still
maps directly to the source code and can be seen as a data
representation of the source text.

The first intermediate representation (IR) constructed from the AST is
the \emph{high-level IR (HIR)}. This is the first IR in that it contains
much less high-level features, because many have been desugared. The HIR
is used to type check the program. In that process an even lower-level
IR is created, the \emph{typed HIR (THIR)}. Still in the form of a tree,
this IR contains the full explicit type information.

The THIR is subsequently used to construct the \emph{mid-level IR (MIR)}
that has the form of a \emph{control-flow graph (CFG)}. Such a graph is
a diagram that consists of \emph{basic blocks} of code and arrows
between them that represent all the possible paths the control flow can
take. The MIR still has generic types but because it is a CFG it is much
more convenient for performing things like liveness analysis,
optimisations, and most importantly, the borrow check of the program.

The last phase of compilation is the \emph{code generation} phase. In
that step the generic code is \emph{monomorphised} i.e.~copied and
output for each type it is instantiated with. Then a special IR for the
compiler backend framework \emph{LLVM} \cite{llvm} is generated. LLVM
will perform much more performance optimisations and finally generate
the assembly code.

\section{System Overview}

\subsection{Design}

\subsubsection{Components}

The Rust-Stainless verification system consists of multiple components,
most of which are implemented as their own Rust crate. Each component
modularly fulfils a step in the overarching goal of taking Rust code,
transforming it for Stainless and getting a verification result back. In
that sense, Rust-Stainless is an alternative frontend to Stainless for
Rust, much like the Rust compiler is an alternative frontend to LLVM.

Rust-Stainless itself has a frontend and a backend.
\lstinline!stainless_frontend! is the frontend crate of the tool and contains
two executables with which the tool is started and which deal with command line
arguments. The subcommand for the Cargo build tool \passthrough{\lstinline!cargo
stainless!} is the most common way of running Rust-Stainless. Internally, it
calls the second stand-alone binary \lstinline!rustc_to_stainless! which runs
the actual frontend of the tool.

The programmer also sees another interface of Rust-Stainless, its
library \lstinline!libstainless!. The library needs to be
imported as \passthrough{\lstinline!extern crate stainless;!} in all
code that is to be verified. It provides the user-facing parts of
Rust-Stainless like the specification macros and some built-in Stainless
types useful for specification: an immutable generic set and map.

The main work of the translation is performed by the
\passthrough{\lstinline!stainless\_extraction!} crate. The frontend
calls the method \passthrough{\lstinline!extract\_crate!} which
retrieves the HIR from the compiler and translates it to Stainless AST
with the help of another crate,
\passthrough{\lstinline!stainless\_data!}. The latter contains the
auto-generated Rust definitions of the Stainless ASTs as well as code to
serialise them to a binary format.

Finally, \passthrough{\lstinline!stainless\_backend!} is responsible for
spawning and interacting with a JVM subprocess of the actual Stainless
executable. The executable consist of the normal Stainless verification
pipeline but with a custom entry-point called
\passthrough{\lstinline!Noxt-Frontend!}. ``Noxt'' stands for \emph{no
extraction} which means it takes as input serialised trees instead of
extracting trees from the Scala compiler output.

\begin{quote}
\TODO{diagram for this}
\end{quote}

\subsubsection{Pipeline}

Knowing the system's components, we can trace the path of an exemplary
program through the pipeline in more detail. This will allow us to
understand some of the design decisions made.

\begin{lstlisting}[
  language=Rust,
  caption={Example Rust program with a post-condition.},
  label={lst:example}
]
extern crate stainless;
use stainless::*;

struct A(i32);

#[post(ret.0 >= 0)]
fn f(a: A) -> A {
  A(a.0 * a.0)
}
\end{lstlisting}

In \RefCode{lst:example}, the Stainless library is added as an external crate
and then imported such that the post-condition specification attribute
\passthrough{\lstinline!\#[post()]!} is available. The attribute is implemented
as a procedural macro\footnote{Procedural macros invoke user-provided code at
compilation-time and hence, allow for more complex transformations of the AST
inside a macro. \cite[section "Procedural Macros"]{rustref}} and will be
expanded to a closure inside the function (\RefCode{lst:example2}).

\begin{lstlisting}[
  language=Rust,
  caption={Desugared post-condition.},
  label={lst:example2}
]
fn f(a: A) -> A {
  #[clippy::stainless::post]
  |a: A, ret: A| -> bool {ret.0 >= 0};
  A(a.0 * a.0)
}
\end{lstlisting}

Rust-Stainless leverages the heavy lifting done by the Rust compiler. The
frontend invokes the compiler via the \passthrough{\lstinline!rustc\_driver!}
library and runs all the phases until all analyses are complete and pass,
otherwise the tool fails with the error given by
\passthrough{\lstinline!rustc!}. In particular, the compiler does the lexing,
parsing, macro expansion and IR construction. Secondly, the compiler type and
borrow checks the program. This way, we achieve our strong assumption from
\RefSec{correctness-claim} because Rust-Stainless only continues if all checks
pass.

Once analysis is complete, the extraction module is invoked on the HIR.
As Rust's compile model prescribes that the HIR is only ever constructed
for the current crate, our toll also only translates the current crate.
We further examine that limitation of our implementation in \RefSec{impl-limitations}.

The extraction traverses all the \emph{items} of the crate, that is top-level
functions, structs, enums, \passthrough{\lstinline!impl!} blocks and their
methods as well as traits. It directly constructs Stainless definitions for
datatypes. For function and method bodies, the extraction works on the THIR,
which is guaranteed to exist as the type check passed. The body of our example
function in THIR can be seen in \RefApp{apx:thir-example}.

The HIR is very well suited for translation to Stainless AST because it is still
in the form of an AST but all the type information is explicit and implicit
features like method calls and dereferences have been resolved or desugared.
That way, the THIR is the representation that is the closest to Stainless AST
which makes translation simpler. The downside of working with the THIR instead
of the MIR is that lifetime resolution is performed on the MIR. Hence, in the
THIR there is no way of knowing the exact point where a binding or reference
ends its lifetime. This is a serious limitation if one needs the lifetimes in
the translation, for example for some other encoding of mutability as explored
in \RefSec{impl-limitations}.


If all the THIR can be translated to Stainless AST without errors,
i.e.~there are no unsupported features in the program, the AST is
serialised to the custom binary format that is used by Stainless'
caching mechanism. At this point, the program consists of a list of
functions, a list of ADTs and a list of classes.

\begin{lstlisting}[language=Scala]
sealed case class A(_0: MutCell[Int])
sealed case class MutCell[T @mutable]((value: T @mutable) @var)
@pure
def f(a: A): A = {
  freshCopy(A(MutCell[Int](a._0.value * a._0.value)))
} ensuring {
  (ret: A) => ret._0.value >= 0
}
\end{lstlisting}

Either the user specified to export the AST in which case the binary
format is simply written to a file, or the Stainless backend is started.
Rust-Stainless spawns a subprocess with the
\passthrough{\lstinline!Noxt-Frontend!} of Stainless and communicates
with it via standard input and output. The subprocess reads the AST from
a temporary file, performs some minor transformation and passes it
through the verification pipeline. In the end, it reports back the
results in JSON format such that Rust-Stainless can print them.

\section{Limitations}
\label{impl-limitations}

This chapter presents how Rust-Stainless is tested and gives some examples of
code that the tool can verify. Furthermore, it will decompose the running time
of a typical execution into its subparts and finish by presenting a code example
related to distributed systems implemented by Informal Systems and verified with
Rust-Stainless.

\section{Benchmarks}

The Rust-Stainless repository contains a test suite of 67 passing (positive) and
12 failing (negative) test code examples of in total around 3000
lines.\footnote{Found under \texttt{stainless\_frontend/tests/[pass|fail]} on
the \texttt{mutable-cells} branch.} The suite is run for every pull request and
every commit to the \texttt{master} branch.

To quantitatively evaluate Rust-Stainless, I collected some statistics of all
the positive test examples, displayed in \autoref{measurements}. Only the
positive examples were used for the measurements because the negative examples
may fail at different stages of the extraction or verification pipeline. Thus,
the time measurement is only representative for the positive examples which
complete the verification pipeline. The used benchmarks amount to 2100 lines of
Rust code (LoC), Stainless generates in total 470 verification conditions (VCs)
and the accumulated running time is just under 6 minutes (Total). Note that the
test suite normally (when run with \texttt{cargo test}) runs multiple tests in
parallel and thus completes in less than 2 minutes on the same
machine.\footnote{All the tests were run on a MacBook Pro with a 2.8~GHz
Quad-Core Intel Core i7 and 16~GB of RAM.}

It is instructive to decompose the running time into its different parts. As
described in \autoref{sec:extraction-overview}, the frontend first needs to
detect some standard library items before it can translate the user crate. Even
if the detection iterates through the same crates on each run, one observes
larger variations in its running time. On average, detection takes $199 \pm
33$~ms or equivalently 4~\% of the running time.

The actual translation is very fast, especially for the rather short examples
that account for the majority of the test suite. It takes only 0.4~ms or about
0.008~\% of the running time on average. By far the largest part of the running
time of our tool is not spent in the frontend itself but rather in the JVM
backend that performs verification. Verification time accounts for on average
95~\% of the total time. The rest is made up of starting, serialising and
reporting. This shows that translation time is inferior to verification time by
multiple orders of magnitudes which is not surprising, given that verification
is much more complex than translation.

\begin{table}
\centering
\caption{Time measurements for all passing test examples of the Rust-Stainless test suite. The presented times are means and standard errors from 5 runs. The tests were performed on the \texttt{mutable-cells} branch, except for the two tests marked with * which are from \texttt{master}. All times are in milliseconds.}
\label{measurements}
\begin{tabular}{lrrrrrr}
\toprule
Name &  LoC &  VCs & Std Item Detection &  Translation &       Verification &              Total \\
\midrule
AdtUseBeforeDeclare      &   13 &    0 &      $170.3\pm7.9$ &  $0.3\pm0.0$ &    $3318.5\pm98.9$ &   $3535.3\pm106.7$ \\
Adts                     &   90 &    3 &      $169.0\pm3.6$ &  $0.5\pm0.1$ &    $4142.0\pm95.8$ &   $4361.7\pm101.2$ \\
Blocks                   &   27 &    2 &      $177.7\pm2.6$ &  $0.4\pm0.1$ &   $3696.6\pm302.0$ &   $3907.0\pm304.2$ \\
Boxes                    &    9 &    1 &      $163.8\pm9.5$ &  $0.3\pm0.0$ &   $3502.3\pm206.8$ &   $3723.5\pm218.2$ \\
CastCorrectness          &   43 &   13 &      $160.6\pm3.8$ &  $0.6\pm0.0$ &   $4683.2\pm253.6$ &   $4926.3\pm257.9$ \\
Clone                    &   37 &    1 &      $158.6\pm6.3$ &  $0.4\pm0.1$ &   $3872.3\pm238.8$ &   $4106.7\pm246.9$ \\
DoubleRefParam           &   24 &    5 &      $170.6\pm7.0$ &  $0.4\pm0.0$ &   $3930.9\pm146.8$ &   $4152.4\pm150.9$ \\
ExternalFn               &   22 &    2 &      $177.4\pm6.1$ &  $0.4\pm0.1$ &   $3768.7\pm181.8$ &   $3981.8\pm188.4$ \\
Fact                     &   14 &    3 &      $181.2\pm5.0$ &  $0.3\pm0.0$ &   $4415.8\pm238.8$ &   $4633.4\pm243.4$ \\
FnRefParam               &   24 &    5 &      $169.8\pm7.2$ &  $0.4\pm0.0$ &   $3937.2\pm189.8$ &   $4158.4\pm198.2$ \\
GenericId                &    9 &    0 &      $178.4\pm3.8$ &  $0.3\pm0.0$ &   $3238.7\pm157.9$ &   $3441.6\pm161.4$ \\
GenericOption            &   37 &   10 &      $178.0\pm3.6$ &  $0.5\pm0.0$ &   $4130.9\pm237.2$ &   $4339.8\pm241.2$ \\
GenericResult            &   31 &    8 &      $169.3\pm8.8$ &  $0.4\pm0.0$ &   $4108.2\pm158.0$ &   $4333.3\pm168.1$ \\
ImplFns                  &   37 &    5 &      $170.9\pm6.7$ &  $0.5\pm0.0$ &   $3973.6\pm169.2$ &   $4205.7\pm175.1$ \\
ImplMutSpec              &   19 &    1 &      $176.3\pm4.3$ &  $0.4\pm0.0$ &   $3878.1\pm277.6$ &   $4089.3\pm282.4$ \\
Implies                  &    9 &    1 &      $177.4\pm4.8$ &  $0.3\pm0.0$ &   $3515.5\pm142.5$ &   $3724.0\pm147.6$ \\
InsertionSort            &  111 &   42 &      $157.2\pm5.7$ &  $0.9\pm0.1$ &   $5961.8\pm218.3$ &   $6236.6\pm225.8$ \\
IntOperators             &  189 &   70 &      $175.1\pm5.7$ &  $0.9\pm0.1$ &  $10136.4\pm380.2$ &  $10384.3\pm385.4$ \\
IntOption                &   30 &    1 &      $167.4\pm2.9$ &  $0.4\pm0.0$ &   $3756.4\pm118.2$ &   $3973.1\pm121.9$ \\
LetType                  &   19 &    2 &      $157.5\pm3.8$ &  $0.4\pm0.1$ &   $3782.8\pm157.2$ &   $4002.8\pm161.9$ \\
ListBinarySearch         &   93 &   23 &      $156.0\pm3.7$ &  $0.8\pm0.0$ &   $6651.8\pm211.2$ &   $6899.7\pm215.8$ \\
MapOps                   &   40 &   31 &      $159.2\pm4.0$ &  $0.6\pm0.0$ &   $5129.7\pm129.9$ &   $5391.5\pm134.8$ \\
Monoid                   &   49 &    9 &      $179.1\pm5.9$ &  $0.5\pm0.0$ &   $6637.3\pm335.7$ &   $6849.5\pm341.3$ \\
MutClone                 &   11 &    1 &      $155.4\pm2.4$ &  $0.3\pm0.0$ &   $3657.1\pm120.1$ &   $3882.0\pm123.3$ \\
MutLocalFields           &   55 &    6 &      $177.2\pm6.7$ &  $0.5\pm0.0$ &   $4352.0\pm224.4$ &   $4563.7\pm230.5$ \\
MutLocalLets             &   29 &    2 &      $156.6\pm4.2$ &  $0.4\pm0.0$ &   $3848.8\pm177.4$ &   $4073.3\pm183.0$ \\
MutLocalParams           &   26 &    3 &      $176.7\pm7.4$ &  $0.4\pm0.0$ &   $3851.8\pm145.1$ &   $4075.6\pm152.6$ \\
MutMemReplace            &   60 &   26 &      $156.6\pm3.9$ &  $0.6\pm0.1$ &   $5905.2\pm194.0$ &   $6147.5\pm200.6$ \\
MutOld                   &   20 &    3 &      $168.8\pm4.6$ &  $0.4\pm0.0$ &   $4006.1\pm161.7$ &   $4232.7\pm165.6$ \\
MutParams                &    8 &    0 &      $177.2\pm4.2$ &  $0.3\pm0.0$ &    $3293.8\pm82.9$ &    $3496.8\pm87.0$ \\
MutRefBorrow0            &   11 &    2 &      $176.7\pm9.2$ &  $0.2\pm0.0$ &   $3659.6\pm151.0$ &   $3876.8\pm160.7$ \\
MutRefBorrow1            &   12 &    1 &      $170.1\pm6.2$ &  $0.3\pm0.0$ &   $3696.8\pm127.2$ &   $3918.7\pm135.0$ \\
MutRefBorrow10           &   10 &    3 &      $166.0\pm6.0$ &  $0.4\pm0.0$ &   $3853.8\pm175.7$ &   $4061.7\pm181.3$ \\
MutRefBorrow11           &   16 &    3 &      $166.3\pm4.4$ &  $0.4\pm0.0$ &   $3980.5\pm113.2$ &   $4189.8\pm118.3$ \\
MutRefBorrow12           &   51 &   27 &      $160.2\pm5.6$ &  $0.7\pm0.1$ &   $5617.2\pm159.7$ &   $5844.9\pm166.5$ \\
MutRefBorrow2            &   13 &    1 &      $168.7\pm6.1$ &  $0.3\pm0.0$ &   $3710.1\pm141.0$ &   $3928.4\pm147.8$ \\
MutRefBorrow3            &   14 &    1 &     $173.5\pm18.3$ &  $0.4\pm0.0$ &   $3771.0\pm219.6$ &   $3995.9\pm238.8$ \\
MutRefBorrow5            &   23 &    3 &      $170.1\pm5.1$ &  $0.4\pm0.0$ &   $3995.0\pm171.1$ &   $4222.1\pm175.3$ \\
MutRefBorrow6            &   26 &    0 &      $165.8\pm5.4$ &  $0.4\pm0.0$ &   $3504.3\pm129.3$ &   $3712.2\pm134.0$ \\
MutRefBorrow7            &   31 &    3 &      $162.6\pm4.3$ &  $0.5\pm0.0$ &    $4108.6\pm98.1$ &   $4337.1\pm104.6$ \\
MutRefBorrow8            &   18 &    3 &      $175.6\pm7.9$ &  $0.3\pm0.0$ &   $3896.0\pm122.6$ &   $4111.9\pm131.4$ \\
MutRefBorrow9            &   17 &    3 &      $174.9\pm7.6$ &  $0.3\pm0.0$ &   $4094.6\pm544.4$ &   $4311.0\pm552.8$ \\
MutRefClone              &   12 &    1 &      $157.0\pm5.2$ &  $0.3\pm0.0$ &   $3685.4\pm167.3$ &   $3912.6\pm174.5$ \\
MutRefImmutBorrow        &   33 &    5 &      $169.6\pm6.3$ &  $0.3\pm0.0$ &   $3955.2\pm184.3$ &   $4175.5\pm191.7$ \\
MutRefTuple              &    9 &    1 &      $175.1\pm4.8$ &  $0.3\pm0.0$ &   $3749.6\pm178.5$ &   $3965.2\pm183.8$ \\
MutReturn                &   51 &   14 &      $157.7\pm5.3$ &  $0.6\pm0.0$ &   $4485.3\pm192.3$ &   $4708.7\pm199.1$ \\
MutTuple                 &    8 &    1 &      $179.8\pm5.5$ &  $0.3\pm0.0$ &   $3712.1\pm113.4$ &   $3917.6\pm118.7$ \\
NestedSpec               &   16 &    2 &      $179.2\pm5.7$ &  $0.3\pm0.0$ &   $3652.8\pm134.4$ &   $3864.9\pm140.1$ \\
NestedSpecImpl           &   20 &    2 &     $172.1\pm11.2$ &  $0.4\pm0.0$ &   $3748.0\pm245.4$ &   $3976.8\pm259.6$ \\
PanicType                &   33 &   15 &      $159.0\pm9.0$ &  $0.5\pm0.1$ &   $4415.7\pm201.3$ &   $4651.8\pm211.1$ \\
PhantomData              &   22 &    2 &      $158.2\pm6.2$ &  $0.4\pm0.0$ &   $3747.4\pm181.1$ &   $3967.0\pm187.3$ \\
ReturnStmt               &   65 &   17 &      $165.9\pm6.9$ &  $0.5\pm0.0$ &   $6032.1\pm149.9$ &   $6269.0\pm158.2$ \\
SetOps                   &   20 &    2 &      $158.3\pm5.6$ &  $0.3\pm0.0$ &    $3577.1\pm81.1$ &    $3815.8\pm87.5$ \\
SpecOnTraitImpl          &   22 &    3 &      $180.2\pm7.9$ &  $0.4\pm0.0$ &   $4434.1\pm172.8$ &   $4650.9\pm180.2$ \\
Strings                  &   39 &    4 &      $164.1\pm9.3$ &  $0.5\pm0.0$ &   $3861.3\pm180.7$ &   $4076.7\pm191.5$ \\
StructUpdate             &   23 &    1 &      $177.7\pm7.7$ &  $0.3\pm0.0$ &   $3766.9\pm204.1$ &   $3974.8\pm212.6$ \\
TraitBounds*             &  106 &   14 &      $159.7\pm5.8$ &  $0.7\pm0.0$ &    $6090.8\pm55.6$ &    $6335.7\pm58.7$ \\
TupleMatch               &   12 &    2 &      $174.7\pm5.7$ &  $0.3\pm0.1$ &    $3752.7\pm21.8$ &    $3969.1\pm28.7$ \\
TupleResult              &    9 &    4 &      $162.7\pm1.9$ &  $0.4\pm0.1$ &    $3822.3\pm20.4$ &    $4027.1\pm20.3$ \\
Tuples                   &   44 &    5 &      $174.2\pm2.9$ &  $0.5\pm0.1$ &    $4054.1\pm33.8$ &    $4281.5\pm33.4$ \\
TypeClass*               &   83 &   28 &      $157.1\pm4.4$ &  $0.7\pm0.0$ &   $11584.7\pm80.2$ &   $11815.9\pm82.0$ \\
TypeClassCallSpecs       &   18 &    2 &      $176.5\pm1.9$ &  $0.4\pm0.0$ &     $3757.2\pm9.2$ &    $3964.7\pm10.4$ \\
TypeClassMultiLookup     &   59 &    8 &      $166.6\pm1.8$ &  $0.6\pm0.0$ &    $5055.9\pm21.8$ &    $5280.2\pm20.5$ \\
TypeClassSpecs           &   28 &    5 &      $177.8\pm1.8$ &  $0.5\pm0.1$ &    $4450.4\pm12.1$ &    $4663.8\pm11.2$ \\
TypeClassWithoutEvidence &   27 &    2 &      $166.6\pm3.2$ &  $0.4\pm0.0$ &    $4409.6\pm27.3$ &    $4626.6\pm29.5$ \\
UseLocal                 &   19 &    6 &      $177.2\pm2.0$ &  $0.5\pm0.0$ &    $4736.6\pm43.2$ &    $4954.7\pm42.1$ \\
UseStd                   &    9 &    0 &      $176.1\pm2.7$ &  $0.2\pm0.0$ &    $3086.8\pm17.7$ &    $3286.5\pm19.4$ \\
\bottomrule
\end{tabular}
\end{table}



\section{Code Examples}

After having quantitatively measured our tool, it is time to qualitatively
evaluate what it is capable of and to that end present some code examples. The
first three examples stem from the test suite of Rust-Stainless and are included
in the measurements from above. The fourth example is the running example from
\autoref{background} and the last one is an implementation by Informal Systems
used in the \texttt{tendermint-rs} repository.

\subsection{Test Suite Examples}

\paragraph{Insertion Sort}

The first example is \autoref{lst:insertion-sort}, an implementation of
\emph{insertion sort} on the already mentioned functional linked-list. The
recursive data type is enabled by the support for boxes. As this example was
translated from an equivalent Scala example in Stainless's test suite, it shows
a functional way of writing Rust; there is no mutation in the entire example. In
idiomatic Rust, one would probably use a vector instead of a linked-list.

The example further showcases the use of implementation blocks and specs on
methods. For proving termination of the recursive implementation it is also
crucial to add the new \lstinline!measure! attributes. Also note that the
\lstinline!Option! type is the one of the standard library.

\paragraph{Type Class}

The next example in \autoref{lst:type-class} is also completely immutable and
demonstrates the use of a trait for equality (\lstinline!Equals!) with two
implementations. Note that one would usually use the traits from the standard
library (\lstinline!Eq! and \lstinline!PartialEq!) and let the macro derive
implementations them. Here however, equality is implemented in the program
because it serves as a good example for the contract one can add on a trait. The
trait contains three laws corresponding to the three properties of the equality
relation.

The implementation for \lstinline!i32! is trivial except that it needs to
dereference the two operands. This is to force the compiler of using the
primitive comparison operator instead of  \lstinline!PartialEq::eq!. The
linked-list implementation (\autoref{lst:equals-list}) is a good example of a
trait bound (\lstinline!T: Equals!) that will be translated to an evidence
argument. The two type class method calls on line 5 will be translated with the
correct receivers.

\begin{lstlisting}[
  language=Rust,
  caption={Equality implemented for the linked-list.},
  label={lst:equals-list}
]
impl<T: Equals> Equals for List<T> {
  fn equals(&self, other: &List<T>) -> bool {
    match (self, other) {
      (List::Nil, List::Nil) => true,
      (List::Cons(x, xs), List::Cons(y, ys)) => x.equals(y) && xs.equals(ys),
      _ => false,
    }
  }
  ...
}
\end{lstlisting}

Lastly, note that the three laws need to be reimplemented in the list
implementation. This is due to Stainless not being able to recursively proof the
properties without some guidance by the programmer. However, the same is
necessary in an equivalent Scala example. Hence, this is not a limitation of the
Rust frontend but of Stainless itself. On the other hand, the law
implementations show another construct that the frontend is able to translate:
the static method calls like \lstinline!Self::law_transitive(xs, ys, zs)! are
correctly translated to calls on the right receiver instances.

\paragraph{Local Mutability}

Turning to mutability, the benchmark in \autoref{lst:local-clone} is available
on the \texttt{mutable-cells} branch. It features a struct for which the macro
derives a \lstinline!Clone! instance. In the function, the struct is created,
cloned and mutated. The assertions ensure that the cloned instance is not
changed as well by the mutation. The mutability translation  will replace the
\lstinline!clone! call with \lstinline!freshCopy!.

\begin{lstlisting}[
  float,
  language=Rust,
  caption={Local mutation and cloning of a simple struct.},
  label=lst:local-clone
]
#[derive(Clone)]
pub struct S(i32);

pub fn main() {
  let mut a = S(1);
  let b = a.clone();
  a.0 = 10;
  assert!(a.0 != b.0)
}
\end{lstlisting}


\paragraph{Mutable References}

The running example \autoref{running-example} from \autoref{background} is
equivalent to the most complicated mutability benchmark on the
\texttt{mutable-cells} branch. In particular, it shows how a mutable reference
is matched upon and then a sub-reference is returned. The sub-reference is later
used to alter the container struct. In summary, the example requires support for
mutable borrows, pattern matching on mutable references, passing mutable
references as values, and struct mutation via references.

In the translation of \lstinline!get_mut_by_id!
(\autoref{lst:mut-ref-translated}), one can observe how the optimisations of
\autoref{optimisations} apply. The \lstinline!id! argument is not wrapped in a
mutable cell because it is immutable. The pattern match binds the mutable cell
\lstinline!v! inside the tuple ADT instead of its value, such that it can later
be used to modify the original object. Furthermore, the evidence argument
responsible for the trait method is added as parameter.

\begin{lstlisting}[
  language=Scala,
  caption={Translation of the \lstinline!get_mut_by_id! method from \autoref{running-example}.},
  label={lst:mut-ref-translated}
]
def get_mut_by_id[K @mutable, V @mutable](
  self: MutCell[Container[K, V]],
  id: Long,
  ev0: Id[K] @evidence
): Option[V] = {
  self.value.pair match {
    case MutCell(Some(MutCell(Tuple2(k, v))))
      if ev0.id(k.value) == id => Some[V](v)
    case _ => None[V]()
  }
} ensuring {
  (ret: Option[V]) => is_empty[K, V](self.value) ==> ret match {
    case None() => true
    case _ => false
  }
}
\end{lstlisting}

\subsection{Peer-list Implementation}
\label{peerlist}

One of the main motivations of this thesis project was to verify some real-world
code examples of Informal Systems. The \lstinline!PeerList! data structure
seemed like the perfect benchmark because it had already some runtime checked
invariants, defined by Romain Rüetschi. Thus, a primary driver of this thesis
was  to verify a version of the structure that was as close to the original code
as possible.

Thanks to the mutability translation and especially the \emph{in-place update}
feature, I achieved that goal adequately, found a problem in the original
specification and was able to strengthen the post-conditions of some methods.
The verified version of the code is in \autoref{apx:peerlist} and by comparing
with  the original one can see, that there are indeed no large
changes.\footnote{The original version can be found here:
\url{https://github.com/informalsystems/tendermint-rs/blob/d8e18c647cd8695d16610c4292b15ec6d1b45fbc/light-client/src/peer_list.rs}}

The peer-list is a data structure that keeps track of different nodes in a
distributed system context. Each node has a current value, recorded in the
\lstinline!values! map, and each node is either the primary node of the system,
a trusted witness, a simple but working full-node or a considered as a faulty
node. These states are recorded with sets in the struct and the
\lstinline!invariant! method returns true if the sets are in a valid
configuration. For example, they all  need to be disjoint as each node is in
exactly one state. The two methods of interest are
\lstinline!replace_faulty_witness! and \lstinline!replace_faulty_primary! that
both  change the states of some of the nodes.

The only notable but crucial change from the original code is that instead of
using the standard \lstinline!HashMap! and  \lstinline!HashSet!, the verified
code uses two custom implementations, \lstinline!ListMap! and
\lstinline!ListSet!. I implemented these two especially for the benchmark and
they are in a separate module in the verified crate (\autoref{apx:listmap}). The
need for list-backed  set and map implementations arose because some methods get
an element from the collections with \lstinline!.iter().next()! in the original
code. Neither does Rust-Stainless support iterators, nor do the map and set from
the Stainless crate support element retrieval. Therefore, I created the
list-backed collections and added a \lstinline!first! method.

Ideally, the two list-collections should be in the Stainless crate as well and
thus, be part of  the library and interface of Rust-Stainless. If that was the
case, the original peer-list example would really only have to import the
Stainless crate and change  the collection types to be verifiable.
Unfortunately,  the one-crate-limitation (cf. \autoref{impl-limitations}) makes
this currently impossible.

Nonetheless, verifying the peer-list implementation with
\passthrough{\lstinline!cargo stainless!} allowed me  to improve the
specifications of the code. First, the pre-condition that the invariant must
hold before the two methods was not set in the code. Furthermore, I could
strengthen the post-conditions of  both methods so as to also prove that the
methods perform the correct changes. For example, \autoref{lst:postcondition}
shows the post-condition of \lstinline!replace_faulty_primary!. If the method
succeeds and returns an \lstinline!Ok! result, the \lstinline!self! not only has
to satisfy the invariant but also that the primary given as argument  was
actually put into the faulty nodes. This avoids that the method does nothing
which would also uphold the invariant. Note also the use of the
\lstinline!old(&self)! helper to use the values before the function in the
specs.

\begin{lstlisting}[
  language=Rust,
  caption={Post-condition of the \lstinline!replace_faulty_primary! method.},
  label={lst:postcondition}
]
#[post((matches!(ret, Ok(_))).implies(
  Self::invariant(&self)
    && old(&self).primary != self.primary
    && self.faulty_nodes.contains(&old(&self).primary)
    && old(&self).witnesses.contains(&self.primary)
))]
\end{lstlisting}

An earlier version of the tool was also capable of verifying an immutable
version  of another example from Informal Systems. The code stems from an
\emph{inter-blockchain communication protocol (IBC)} handler implementation and
was rewritte to be immutable and self-contained.\footnote{The PR with the
example can be found here:
\url{https://github.com/informalsystems/ibc-rs/pull/759}.}

\hfill \break \noindent With five different examples showcasing each their own
set of language features, this chapter showed what Rust-Stainless can currently
process. It became clear that no matter how complicated the translations the
frontend performs are, the running time is dominated by the verification time of
the backend. For an overview of the tool's limitations, see
\autoref{impl-limitations}. The next chapter, summarises research  work in
related areas like ownership systems and formal verification.

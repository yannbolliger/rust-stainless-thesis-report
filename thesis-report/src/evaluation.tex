In this chapter, we present how Rust-Stainless is tested and give some examples
of code that it can process. Furthermore, we will decompose the running time of
a typical execution into its subparts and finish by presenting a
blockchain-related code example implemented by Informal Systems and verified
with Rust-Stainless.

\section{Benchmarks}

The Rust-Stainless repository contains a test suite of 64 passing (positive) and
11 failing (negative) test code examples of in total around 3000
lines.\footnote{Found under \texttt{stainless\_frontend/tests/[pass|fail]}
on the \texttt{mutable-cells} branch.} The suite is run for every pull request
and every commit to the \texttt{master} branch.

To quantitatively evaluate Rust-Stainless, we collected some statistics of all
the positive examples and display them in \autoref{measurements}.\footnote{All
the tests were run on a MacBook Pro with a 2.8~GHz Quad-Core Intel Core i7 and
16~GB of RAM.} Only the positive examples were used for the measurements because
the negative examples may fail at different stages of the extraction or
verification pipeline. Hence, the time measurement is only representative for
the positive examples which complete the verification pipeline. The displayed
benchmarks amount to 2100 lines of Rust code (LoC), Stainless generates in total
470 verification conditions (VCs) and the accumulated running time is just under
6 minutes (Total). Note that the test suite normally (when run with
\texttt{cargo test}) runs multiple tests in parallel and thus completes in less
than 2 minutes on the same machine.

It is instructive to decompose the running time of the test examples into the
different parts of the pipeline. As described in
\autoref{sec:extraction-overview}, the frontend first needs to detect some
standard library items before it can translate the user crate. This detection is
needed because the numerical identifiers used in the compilation
(\lstinline!DefId!s) are not stable across runs. Even if the detection iterates
through the same constant number of crates on each run, we observe larger
variations in running time. On average in the measured test examples, detection
takes $200 \pm 32$~ms or equivalently 4~\% of the running time.

The actual translation is very fast, especially for the rather short examples
that are the majority of the test suite, it takes only 0.4~ms or about 0.008~\%
of the running time on average. By far the most expensive part of our pipeline
is not spent in the frontend itself but rather the JVM Stainless backend that
performs verification. Verification time accounts for on average 95~\% of the
total running time. The rest of the time is made up of starting, serialising and
reporting. This shows that translation time is inferior to verification time by
1-2 orders of magnitudes which is not surprising, given that verification is
much more complex than our translation.

\begin{table}
\centering
\caption{Time measurements for all passing test examples of the Rust-Stainless test suite. The presented times are means and standard errors from 5 runs. The tests were performed on the \texttt{mutable-cells} branch, except for the two tests marked with * which are from \texttt{master}. All times are in milliseconds.}
\label{measurements}
\begin{tabular}{lrrrrrr}
\toprule
Name &  LoC &  VCs & Std Item Detection &  Translation &       Verification &              Total \\
\midrule
AdtUseBeforeDeclare      &   13 &    0 &      $170.3\pm7.9$ &  $0.3\pm0.0$ &    $3318.5\pm98.9$ &   $3535.3\pm106.7$ \\
Adts                     &   90 &    3 &      $169.0\pm3.6$ &  $0.5\pm0.1$ &    $4142.0\pm95.8$ &   $4361.7\pm101.2$ \\
Blocks                   &   27 &    2 &      $177.7\pm2.6$ &  $0.4\pm0.1$ &   $3696.6\pm302.0$ &   $3907.0\pm304.2$ \\
Boxes                    &    9 &    1 &      $163.8\pm9.5$ &  $0.3\pm0.0$ &   $3502.3\pm206.8$ &   $3723.5\pm218.2$ \\
CastCorrectness          &   43 &   13 &      $160.6\pm3.8$ &  $0.6\pm0.0$ &   $4683.2\pm253.6$ &   $4926.3\pm257.9$ \\
Clone                    &   37 &    1 &      $158.6\pm6.3$ &  $0.4\pm0.1$ &   $3872.3\pm238.8$ &   $4106.7\pm246.9$ \\
DoubleRefParam           &   24 &    5 &      $170.6\pm7.0$ &  $0.4\pm0.0$ &   $3930.9\pm146.8$ &   $4152.4\pm150.9$ \\
ExternalFn               &   22 &    2 &      $177.4\pm6.1$ &  $0.4\pm0.1$ &   $3768.7\pm181.8$ &   $3981.8\pm188.4$ \\
Fact                     &   14 &    3 &      $181.2\pm5.0$ &  $0.3\pm0.0$ &   $4415.8\pm238.8$ &   $4633.4\pm243.4$ \\
FnRefParam               &   24 &    5 &      $169.8\pm7.2$ &  $0.4\pm0.0$ &   $3937.2\pm189.8$ &   $4158.4\pm198.2$ \\
GenericId                &    9 &    0 &      $178.4\pm3.8$ &  $0.3\pm0.0$ &   $3238.7\pm157.9$ &   $3441.6\pm161.4$ \\
GenericOption            &   37 &   10 &      $178.0\pm3.6$ &  $0.5\pm0.0$ &   $4130.9\pm237.2$ &   $4339.8\pm241.2$ \\
GenericResult            &   31 &    8 &      $169.3\pm8.8$ &  $0.4\pm0.0$ &   $4108.2\pm158.0$ &   $4333.3\pm168.1$ \\
ImplFns                  &   37 &    5 &      $170.9\pm6.7$ &  $0.5\pm0.0$ &   $3973.6\pm169.2$ &   $4205.7\pm175.1$ \\
ImplMutSpec              &   19 &    1 &      $176.3\pm4.3$ &  $0.4\pm0.0$ &   $3878.1\pm277.6$ &   $4089.3\pm282.4$ \\
Implies                  &    9 &    1 &      $177.4\pm4.8$ &  $0.3\pm0.0$ &   $3515.5\pm142.5$ &   $3724.0\pm147.6$ \\
InsertionSort            &  111 &   42 &      $157.2\pm5.7$ &  $0.9\pm0.1$ &   $5961.8\pm218.3$ &   $6236.6\pm225.8$ \\
IntOperators             &  189 &   70 &      $175.1\pm5.7$ &  $0.9\pm0.1$ &  $10136.4\pm380.2$ &  $10384.3\pm385.4$ \\
IntOption                &   30 &    1 &      $167.4\pm2.9$ &  $0.4\pm0.0$ &   $3756.4\pm118.2$ &   $3973.1\pm121.9$ \\
LetType                  &   19 &    2 &      $157.5\pm3.8$ &  $0.4\pm0.1$ &   $3782.8\pm157.2$ &   $4002.8\pm161.9$ \\
ListBinarySearch         &   93 &   23 &      $156.0\pm3.7$ &  $0.8\pm0.0$ &   $6651.8\pm211.2$ &   $6899.7\pm215.8$ \\
MapOps                   &   40 &   31 &      $159.2\pm4.0$ &  $0.6\pm0.0$ &   $5129.7\pm129.9$ &   $5391.5\pm134.8$ \\
Monoid                   &   49 &    9 &      $179.1\pm5.9$ &  $0.5\pm0.0$ &   $6637.3\pm335.7$ &   $6849.5\pm341.3$ \\
MutClone                 &   11 &    1 &      $155.4\pm2.4$ &  $0.3\pm0.0$ &   $3657.1\pm120.1$ &   $3882.0\pm123.3$ \\
MutLocalFields           &   55 &    6 &      $177.2\pm6.7$ &  $0.5\pm0.0$ &   $4352.0\pm224.4$ &   $4563.7\pm230.5$ \\
MutLocalLets             &   29 &    2 &      $156.6\pm4.2$ &  $0.4\pm0.0$ &   $3848.8\pm177.4$ &   $4073.3\pm183.0$ \\
MutLocalParams           &   26 &    3 &      $176.7\pm7.4$ &  $0.4\pm0.0$ &   $3851.8\pm145.1$ &   $4075.6\pm152.6$ \\
MutMemReplace            &   60 &   26 &      $156.6\pm3.9$ &  $0.6\pm0.1$ &   $5905.2\pm194.0$ &   $6147.5\pm200.6$ \\
MutOld                   &   20 &    3 &      $168.8\pm4.6$ &  $0.4\pm0.0$ &   $4006.1\pm161.7$ &   $4232.7\pm165.6$ \\
MutParams                &    8 &    0 &      $177.2\pm4.2$ &  $0.3\pm0.0$ &    $3293.8\pm82.9$ &    $3496.8\pm87.0$ \\
MutRefBorrow0            &   11 &    2 &      $176.7\pm9.2$ &  $0.2\pm0.0$ &   $3659.6\pm151.0$ &   $3876.8\pm160.7$ \\
MutRefBorrow1            &   12 &    1 &      $170.1\pm6.2$ &  $0.3\pm0.0$ &   $3696.8\pm127.2$ &   $3918.7\pm135.0$ \\
MutRefBorrow10           &   10 &    3 &      $166.0\pm6.0$ &  $0.4\pm0.0$ &   $3853.8\pm175.7$ &   $4061.7\pm181.3$ \\
MutRefBorrow11           &   16 &    3 &      $166.3\pm4.4$ &  $0.4\pm0.0$ &   $3980.5\pm113.2$ &   $4189.8\pm118.3$ \\
MutRefBorrow12           &   51 &   27 &      $160.2\pm5.6$ &  $0.7\pm0.1$ &   $5617.2\pm159.7$ &   $5844.9\pm166.5$ \\
MutRefBorrow2            &   13 &    1 &      $168.7\pm6.1$ &  $0.3\pm0.0$ &   $3710.1\pm141.0$ &   $3928.4\pm147.8$ \\
MutRefBorrow3            &   14 &    1 &     $173.5\pm18.3$ &  $0.4\pm0.0$ &   $3771.0\pm219.6$ &   $3995.9\pm238.8$ \\
MutRefBorrow5            &   23 &    3 &      $170.1\pm5.1$ &  $0.4\pm0.0$ &   $3995.0\pm171.1$ &   $4222.1\pm175.3$ \\
MutRefBorrow6            &   26 &    0 &      $165.8\pm5.4$ &  $0.4\pm0.0$ &   $3504.3\pm129.3$ &   $3712.2\pm134.0$ \\
MutRefBorrow7            &   31 &    3 &      $162.6\pm4.3$ &  $0.5\pm0.0$ &    $4108.6\pm98.1$ &   $4337.1\pm104.6$ \\
MutRefBorrow8            &   18 &    3 &      $175.6\pm7.9$ &  $0.3\pm0.0$ &   $3896.0\pm122.6$ &   $4111.9\pm131.4$ \\
MutRefBorrow9            &   17 &    3 &      $174.9\pm7.6$ &  $0.3\pm0.0$ &   $4094.6\pm544.4$ &   $4311.0\pm552.8$ \\
MutRefClone              &   12 &    1 &      $157.0\pm5.2$ &  $0.3\pm0.0$ &   $3685.4\pm167.3$ &   $3912.6\pm174.5$ \\
MutRefImmutBorrow        &   33 &    5 &      $169.6\pm6.3$ &  $0.3\pm0.0$ &   $3955.2\pm184.3$ &   $4175.5\pm191.7$ \\
MutRefTuple              &    9 &    1 &      $175.1\pm4.8$ &  $0.3\pm0.0$ &   $3749.6\pm178.5$ &   $3965.2\pm183.8$ \\
MutReturn                &   51 &   14 &      $157.7\pm5.3$ &  $0.6\pm0.0$ &   $4485.3\pm192.3$ &   $4708.7\pm199.1$ \\
MutTuple                 &    8 &    1 &      $179.8\pm5.5$ &  $0.3\pm0.0$ &   $3712.1\pm113.4$ &   $3917.6\pm118.7$ \\
NestedSpec               &   16 &    2 &      $179.2\pm5.7$ &  $0.3\pm0.0$ &   $3652.8\pm134.4$ &   $3864.9\pm140.1$ \\
NestedSpecImpl           &   20 &    2 &     $172.1\pm11.2$ &  $0.4\pm0.0$ &   $3748.0\pm245.4$ &   $3976.8\pm259.6$ \\
PanicType                &   33 &   15 &      $159.0\pm9.0$ &  $0.5\pm0.1$ &   $4415.7\pm201.3$ &   $4651.8\pm211.1$ \\
PhantomData              &   22 &    2 &      $158.2\pm6.2$ &  $0.4\pm0.0$ &   $3747.4\pm181.1$ &   $3967.0\pm187.3$ \\
ReturnStmt               &   65 &   17 &      $165.9\pm6.9$ &  $0.5\pm0.0$ &   $6032.1\pm149.9$ &   $6269.0\pm158.2$ \\
SetOps                   &   20 &    2 &      $158.3\pm5.6$ &  $0.3\pm0.0$ &    $3577.1\pm81.1$ &    $3815.8\pm87.5$ \\
SpecOnTraitImpl          &   22 &    3 &      $180.2\pm7.9$ &  $0.4\pm0.0$ &   $4434.1\pm172.8$ &   $4650.9\pm180.2$ \\
Strings                  &   39 &    4 &      $164.1\pm9.3$ &  $0.5\pm0.0$ &   $3861.3\pm180.7$ &   $4076.7\pm191.5$ \\
StructUpdate             &   23 &    1 &      $177.7\pm7.7$ &  $0.3\pm0.0$ &   $3766.9\pm204.1$ &   $3974.8\pm212.6$ \\
TraitBounds*             &  106 &   14 &      $159.7\pm5.8$ &  $0.7\pm0.0$ &    $6090.8\pm55.6$ &    $6335.7\pm58.7$ \\
TupleMatch               &   12 &    2 &      $174.7\pm5.7$ &  $0.3\pm0.1$ &    $3752.7\pm21.8$ &    $3969.1\pm28.7$ \\
TupleResult              &    9 &    4 &      $162.7\pm1.9$ &  $0.4\pm0.1$ &    $3822.3\pm20.4$ &    $4027.1\pm20.3$ \\
Tuples                   &   44 &    5 &      $174.2\pm2.9$ &  $0.5\pm0.1$ &    $4054.1\pm33.8$ &    $4281.5\pm33.4$ \\
TypeClass*               &   83 &   28 &      $157.1\pm4.4$ &  $0.7\pm0.0$ &   $11584.7\pm80.2$ &   $11815.9\pm82.0$ \\
TypeClassCallSpecs       &   18 &    2 &      $176.5\pm1.9$ &  $0.4\pm0.0$ &     $3757.2\pm9.2$ &    $3964.7\pm10.4$ \\
TypeClassMultiLookup     &   59 &    8 &      $166.6\pm1.8$ &  $0.6\pm0.0$ &    $5055.9\pm21.8$ &    $5280.2\pm20.5$ \\
TypeClassSpecs           &   28 &    5 &      $177.8\pm1.8$ &  $0.5\pm0.1$ &    $4450.4\pm12.1$ &    $4663.8\pm11.2$ \\
TypeClassWithoutEvidence &   27 &    2 &      $166.6\pm3.2$ &  $0.4\pm0.0$ &    $4409.6\pm27.3$ &    $4626.6\pm29.5$ \\
UseLocal                 &   19 &    6 &      $177.2\pm2.0$ &  $0.5\pm0.0$ &    $4736.6\pm43.2$ &    $4954.7\pm42.1$ \\
UseStd                   &    9 &    0 &      $176.1\pm2.7$ &  $0.2\pm0.0$ &    $3086.8\pm17.7$ &    $3286.5\pm19.4$ \\
\bottomrule
\end{tabular}
\end{table}



\section{Code Examples}

After having measured our tool, we will now quantitatively evaluate what it is
capable of and to that end present some code benchmarks. The first three
examples stem from the test suite of Rust-Stainless and were included in the
previous section's measurements. The last example is an implementation by
Informal Systems used in the IBC implementation.

\subsubsection{Test Suite Examples}

\paragraph{Insertion Sort}

The first code example in \autoref{lst:insertion-sort} is an implementation of
\emph{inseration sort} on the typical functional linked-list seen in
\autoref{lst:linked-list}. The linked-list is implemented as a recursive
datatype which is enabled by the support for boxes. As this example was
translated  from an equivalent Scala example in Stainless' test suite, it shows
a functional way of writing Rust; there is no mutation in the entire example. In
idiomatic Rust, one would probably use a vector instead of a linked-list. On the
other hand, the linked-list allows for a recursive implementation which is very
well-suited for verification with Stainless.

The example further showcases the use of implementation blocks and specs on
implementation methods. For proving termination of the recursive implementation
it is also crucial to add the new \lstinline!measure! attributes. Also note that
the \lstinline!Option! type of  the standard library is used and correctly
translated.

\begin{lstlisting}[
    language=Rust,
    caption={Functional linked-list as a recursive Rust struct.},
    label=lst:linked-list
]
pub enum List<T> {
  Nil,
  Cons(T, Box<List<T>>),
}
\end{lstlisting}

\paragraph{Type Class}

The next example in \autoref{lst:type-class} is still completely immutable and
demonstrates the use of a trait for equality (\lstinline!Equals!) with two
implementations. Note that there are traits for equality in the standard library
(\lstinline!Eq! and \lstinline!PartialEq!), usually one would use that trait and
let the compiler derive implementations for it. Here however, we implement
equality ourselves because it serves as a good example for the algebraic
properties we can add on a trait/type class. The trait contains one abstract
method, one default method (\lstinline!not_equals!) as well as three laws
corresponding to the three properties of the equality relation.

The implementation for \lstinline!i32! is trivial except that it needs to
dereference the two operands. This is to force the compiler of using the
primitive comparison operator instead of desugaring a call to
\lstinline!PartialEq::eq!. The linked-list implementation
(\autoref{lst:equals-list}) is a good example of a trait bound on a type
parameter (\lstinline!T: Equals!) that will be translated to an evidence
argument on the Scala type class implementation (see
\autoref{lst:equals-scala}). The two type class method calls on line 5 will be
translated with the correct receivers, the evidence argument for the first one
and \lstinline!this! for the second one.

\begin{lstlisting}[
  language=Rust,
  caption={Equality implemented for the linked-list.},
  label={lst:equals-list}
]
impl<T: Equals> Equals for List<T> {
  fn equals(&self, other: &List<T>) -> bool {
    match (self, other) {
      (List::Nil, List::Nil) => true,
      (List::Cons(x, xs), List::Cons(y, ys)) => x.equals(y) && xs.equals(ys),
      _ => false,
    }
  }
  ...
}
\end{lstlisting}

Lastly, we have to note that the three laws need to be reimplemented in the list
implementation. This is due to Stainless not being able to recursively proof the
properties without some guidance of the programmer. The same is necessary in an
equivalent Scala example, therefore this is not a limitation of the Rust
frontend but of Stainless itself. The law implementations show however another
construct that the frontend is able to translate correctly: the static method
calls like \lstinline!Self::law_transitive(xs, ys, zs)! are correctly translated
and called on the corresponding type class implementations.

\paragraph{Local Mutability}

Turning to mutability, the benchmark in \autoref{lst:local-clone} is available
on the \texttt{mutable-cells} branch. It features a simple struct containing an
integer for which the compiler derives an instance of the \lstinline!Clone!
trait. In the function, the struct is created cloned and mutated. The assertions
ensure that  the cloned instance is not changed as well by the mutation. The
mutability translation for that example will erase the instance of
\lstinline!Clone! and replace the \lstinline!clone! call with
\lstinline!freshCopy!.

\begin{lstlisting}[
  language=Rust,
  caption={Local mutation and cloning of a simple struct.},
  label=lst:local-clone
]
extern crate stainless;

#[derive(Clone)]
pub struct S(i32);

pub fn main() {
  let mut a = S(1);
  let b = a.clone();
  a.0 = 10;
  assert!(a.0 != b.0)
}
\end{lstlisting}


\paragraph{Mutable References}

A more involved example is \autoref{lst:mut-ref-12} from the
\texttt{mutable-cells} branch. This is the most complicated mutability
benchmark, unfortunately it currently fails verification due to a bug in the
Stainless backend. In particular, the example shows how a mutable reference is
matched upon and then a sub-reference is returned (\autoref{lst:mut-ref-match}).
That sub-reference is subsequently used to alter the container struct. The
assertion at the end of the main function ensures that the mutation has an
effect. In summary, the example requires support for mutable borrows, pattern
match on mutable references, passing mutable references as values in arguments
and returns, as well as mutation of a struct via a mutable reference.

\begin{lstlisting}[
  language=Rust,
  caption={Pattern match on a mutable reference.},
  label={lst:mut-ref-match}
]
impl<V> Container<String, V> {
  pub fn get_mut(&mut self, key: &String) -> Option<&mut V> {
    match &mut self.pair {
      Some((k, v)) if *k == *key => Some(v),
      _ => None,
    }
  }
}
\end{lstlisting}

In the translation of \autoref{lst:mut-ref-match}
(\autoref{lst:mut-ref-translated}), we observe how the optimisations of
\autoref{optimisations} apply. The string argument \lstinline!key! is not given
as a mutable cell because it is immutable. We also see that the pattern match
binds the mutable cell \lstinline!v! inside the tuple ADT instead of its value,
such that it can later be used to modify the original object.

\begin{lstlisting}[
  language=Scala,
  caption={Translation of \autoref{lst:mut-ref-match}},
  label={lst:mut-ref-translated}
]
def get_mut[V @mutable](
  self: MutCell[Container[String, V @mutable]],
  key: String
): Option[V @mutable] =
  self.value.pair match {
    case MutCell(Some(MutCell(Tuple2(k, v)))) if k.value == key =>
      Some[V @mutable](v)
    case _ =>
      None[V @mutable]()
  }
\end{lstlisting}

\subsubsection{Peer-list Implementation}
benchmarks of Informal

\begin{itemize}
  \item show list map \& set
  \item found problem
  \item strengthened post-conditions
\end{itemize}

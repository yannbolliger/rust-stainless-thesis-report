Everyone who has ever written a computer program knows, it never runs correctly
on the first try. Writing correct software is a hard task, yet the longer the
more, software dominates our lives. While finding bugs in a student software
project may be time-consuming but harmless, there are countless places where
logical errors and worse, security problems cannot be tolerated. These are
usually systems where failure has too high of a cost in terms of money or even
human safety, imagine a railway control system. Software errors are also
intolerable in systems that are costly or impossible to upgrade like embedded
systems or satellites.

One type of systems that have both a high financial cost of failure and are
difficult to upgrade are blockchains that power cryptocurrencies. Bitcoin
\cite{bitcoin} and Ethereum \cite{ethereum} have become very popular and are
valued at amounts reaching into the hundreds of billions of dollars at the time
of writing. Upgrading such distributed systems is hard because a majority of the
participating machines need to reach a consensus about the upgrade. This makes
upgrading the two former blockchains nearly impossible and even systems that
were designed with upgrades in mind like the Cosmos network \cite{cosmos} wish
to minimise the number of upgrades required, hence the need for correct
software.

A powerful and strict approach to writing correct software is to mathematically
prove the correctness of a program, called \emph{formal verification}. The
Stainless verification framework \cite{stainless} is a formal verification tool
for the Scala programming language. It enables programmers to prove high-level
properties about functions and data structures, for instance that they conform
to their specification, in a semi-automated fashion. Additionally, Stainless
establishes program termination and the absence of runtime crashes.

However, blockchains are usually not implemented in Scala, but rather in
C\texttt{++}, Go or Rust for performance reasons. While Go achieves high
performance despite its \emph{garbage collector}, C\texttt{++} and Rust leave
memory management to the programmer. This is the primary source of security
problems in C\texttt{++}. Rust however guarantees memory safety by introducing a
new type checking phase in its compiler, the \emph{borrow checker}. Rust is
therefore well-suited for implementing highly performant, correct, and safe
systems, for example the \emph{Tendermint blockchain consensus}
implementation\footnote{\url{https://github.com/informalsystems/tendermint-rs}}
developed by Informal Systems.\footnote{\url{https://informal.systems}} But even
with type safety and memory safety at compile-time, Rust cannot guarantee
correctness on its own.

Therefore, and with the vision of combining the safety guarantees of Rust with
the formally verified correctness guarantees of Stainless, we present
\emph{Rust-Stainless}, a verification tool created by Georg Schmid.
Rust-Stainless\footnote{Pun not intended, but -- needless to say -- welcome.} is
a frontend to the Stainless verifier capable of extracting a subset of the Rust
language, translating it to a subset of Scala and verifying it with Stainless.
In this thesis project, I substantially extend the fragment of Rust that can be
translated by the tool, adding features like mutability, references, and type
classes. \newpage

Before this thesis project, I mainly worked on the translation of Rust traits to
Scala type classes, in the context of a semester project supervised by Georg
Schmid. The thesis project was the best way to continue  the ongoing work on
Rust-Stainless. Therefore, this report also presents some of the results of the
previous project. Thanks to Georg and Prof. Viktor Kun\v{c}ak, my academic
supervisor, I had the luck to conduct this thesis project as an intern at
Informal Systems, where I was supervised by Romain Rüetschi. The blockchain- and
Rust-focussed company provided the perfect motivation and context to push
Rust-Stainless in the right direction.

\subsubsection{Contributions}

\begin{itemize}

\item \textbf{Theory} In \autoref{translation}, I develop a translation from
Rust to Scala that produces equivalent runtime semantics in Scala for Rust
features like mutability, mutable and immutable references, mutable tuples and
data types, and move semantics. The translation is further adapted for use with
the current state of Stainless's imperative phase.

\item \textbf{Implementation} The largest contribution of this project are the
numerous features that I added  to the implementation of Rust-Stainless. In over
70 pull requests, I fixed bugs, added extraction capabilities for new Rust
language fragments, implemented the mutability translation, and improved the
user-facing \lstinline!stainless! library. Chapter \ref{implementation} first
describes the overall design and pipeline of the tool, then it outlines the
state of the frontend before and after this project.

\item \textbf{Bugfixes} By using Stainless only as a backend and with unforeseen
Scala-atypical inputs, we uncovered 14 issues in Stainless of which I solved or
helped solve eight. Furthermore, I added the \lstinline!freshCopy! primitive to
the imperative phase of Stainless.

\item \textbf{User Perspective} The internship at Informal Systems allowed me to
test our tool on real-world code, see \autoref{peerlist}. This helped taking on
a user perspective and had great influence on the choice and priorisation of new
features.

\end{itemize}

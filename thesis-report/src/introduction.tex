Everyone who has ever written a computer program knows, it never runs correctly
at the first try. Writing correct software is a hard task, yet software
dominates our lives the longer the more. While finding bugs in a student
software project may be time-consuming but harmless, there are countless places
where logical errors and worse, security problems cannot be tolerated. These are
usually systems where failure has too high of a cost in terms of money or even
human safety, think a railway control system. Software errors are also
intolerable in systems that or costly or impossible to \emph{upgrade} like
embedded systems or satellites.

One type of system that has both a high financial cost of failure and is
difficult to upgrade are blockchains that power decentralised cryptocurrencies.
Bitcoin \cite{bitcoin} and Ethereum \cite{ethereum} have become very popular and
are valued at amounts reaching into the hundreds of billion dollars at the time
of writing. Upgrading such distributed systems is hard because a majority of the
participating machines needs to reach a consensus about the upgrade. This makes
upgrading the two former blockchains nearly impossible and even systems that
were designed with upgrades in mind like the Cosmos network \cite{cosmos} wish
to minimise the number of upgrades required, hence the need for correct
(blockchain) software.

A powerful and strict approach to writing correct software is to mathematically
prove the program correct in an automated fashion, called \emph{formal
verification}. The Stainless verification framework \cite{stainless} is a formal
verification tool for the Scala programming language. It enables programmers to
prove high-level properties about functions and data structures, for instance
that they conform to their specification. Additionally, it establishes program
termination and the absence of runtime crashes.

However, blockchains are usually not implemented in Scala, but rather in C++, Go
or Rust for performance reasons. While Go achieves high performance despite its
\emph{garbage collector}, C++ and Rust leave memory management to the
programmer. For C++ this is the primary source of security problems, Rust
however guarantees memory safety by introducing a new type checking phase in its
compiler, the \emph{borrow checker}. Therefore, Rust is well-suited for
implementing highly performant, correct and safe blockchain systems, for example
the  the \emph{Tendermint blockchain consensus}
implementation\footnote{\url{https://github.com/informalsystems/tendermint-rs}}
developed by Informal Systems.\footnote{\url{https://informal.systems}} But even
with type safety and memory safety at compile-time, Rust cannot guarantee
correctness on its own.

Therefore, we present \emph{Rust-Stainless}, a verification tool created by
Georg Schmid, with the vision of combining the safety guarantees of Rust with
formally verified correctness guarantees of Stainless.
Rust-Stainless\footnote{Pun not intended, but -- needless to say -- welcome.} is
a frontend to the Stainless verifier capable of extracting a subset of the Rust
language, translating it to a subset of Scala and verifying it with Stainless.
In this thesis project, I substantially extend the fragment of Rust that can be
translated by the tool including features like mutability, references and type
classes.

\subsubsection{Contributions}

\begin{itemize}

\item \textbf{Theory} In \autoref{translation}, I develop a translation from
Rust to Scala that produces equivalent runtime semantics in Scala for Rust
features like mutability, mutable and immutable references, mutable tuples and
data types, and move semantics. The translation is further adapted for use with
the current state of Stainless's imperative phase.

\item \textbf{Implementation} The largest contribution of this project are the
numerous features that I added  to the implementation of the Rust-Stainless
tool. In over 70 pull requests, I fixed bugs, added extraction capabilities for
new Rust language fragments, implemented the mutability translation, and
improved the user-facing \lstinline!stainless! library. Chapter
\ref{implementation} first describes the overall design and pipeline of our
tool. Then it establishes the state of the frontend before and after this thesis
project.

\item \textbf{Bugfixes} By using Stainless only as a backend and with unforeseen
Scala-atypical inputs, we uncovered 14 issues in Stainless, from which I solved
or helped solve eight. Furthermore, I added the \lstinline!freshCopy! primitive
to the imperative phase of Stainless.

\item \textbf{User Perspective} The internship at Informal Systems allowed me to
test our tool on real-world code, see \autoref{peerlist}. This helped taking on
a user perspective and had great influence on the choice and priorisation of new
features.

\end{itemize}

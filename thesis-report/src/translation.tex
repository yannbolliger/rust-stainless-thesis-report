Although Rust introduces many ideas from functional programming into systems
programming, it nevertheless is a language for high performance and low-level
programming. Therefore, \emph{idiomatic Rust} heavily uses mutability. To avoid
having to rewrite Rust source code before being able to verify it with
Rust-Stainless, the tool needs to support mutability. In this chapter, I
introduce a translation of Rust's owned types and references to Scala that
preserves runtime semantics. Further, I argue that the translation is correct
and then present some changes and optimisations that make it possible to use the
translation together with the current state of the Stainless verifier backend.

\section{Translation for Runtime Equivalence}
\label{sec:translation}

The translation works under the assumption that the input Rust code is from the
\emph{safe subset} \cite[section ``Unsafety'']{rustref} of the 2018 edition of
Rust and that execution is single-threaded. Further, the translation requires
the absence of \emph{interior mutability} \cite[section ``Interior
Mutability'']{rustref}. That means, all types are immutable through shared
references. This can be enforced by prohibiting the use of \emph{smart pointers}
like the ones from \passthrough{\lstinline!std::cell!}. Moreover, the safe
subset of Rust excludes features like raw pointers,
\passthrough{\lstinline!union!}, unsafe functions and blocks, as well as
external code, like linked C-code. In practice, all unsafe features are detected
at extraction of the code and abort the translation.


\subsection{Algorithm}

The goal of the translation is to bridge the gaps between Rust's and Scala's
memory model -- making references accessible to the programer and allowing
mutability of all struct fields if needed. To achieve the first one and bring
the ability of manipulating references to Scala, I introduce a special wrapper
case class, the \emph{mutable cell}:

\begin{lstlisting}[language=Scala, style=short]
case class MutCell[T](var value: t)
\end{lstlisting}

The mutable cell object is used as a layer of indirection on all Scala values
and directly models the \emph{place} of a value in Rust, where the value is
stored in the field (\lstinline!value!). Note that the field is mutable, which
in principle allows changes to any place. After the translation, the mutable
cell's field will even be the only mutable variable. All mutability in the
original Rust program will be modelled with that field.

\begin{quote}
Whenever something is (possibly) mutable, we wrap it into a mutable cell.
\end{quote}

Local variables are wrapped into mutable cells, regardless of their mutability
in Rust. This enables taking references to local variables which is needed for
both mutable and immutable variables. The approach also allows to create
references to data of primitive types like integers, something that is
impossible to do in Scala directly.

Likewise, all \lstinline!struct!, \lstinline!enum!, and tuple fields are mutable
cells with the type of the original field as the type argument of the cell.
Thereby, the translation achieves its second goal from above -- all fields are
\emph{possibly} mutable. In other words, every place in the Rust program is
modelled by the introduction of a mutable cell object. Hence, all place
expressions in the original program correspond to exactly one mutable cell
instance in the translation.

\noindent\begin{minipage}[t]{.45\textwidth}
\begin{lstlisting}[language=Rust, caption={Example Rust struct.}]
struct A<T> {
  a: T, b: i32
}
let x = A {
  a: "foo", b: 123
}
let mut y = 123
assert!(y == x.b)
\end{lstlisting}
\end{minipage}\hfill
\begin{minipage}[t]{.52\textwidth}
\begin{lstlisting}[
  language=Scala,
  caption=The fields of the struct are modelled with mutable
  cells as are the two let-bindings.
]
case class A[T](
  a: MutCell[T], b: MutCell[Int]
)
val x = MutCell(
  A(MutCell("foo"), MutCell(123))
)
val y = MutCell(123)
assert(y.value == x.value.b.value)
\end{lstlisting}
\end{minipage}


\subsubsection{Tuples}

As for all structs, the mutability of tuples is decided at binding time of the
instance in Rust. Contrary to Scala, where tuples are always immutable. To allow
for tuple mutation in Rust, tuples of any positive arity are translated as case
classes instead of Scala tuples. The following example shows  such a class for a
2-tuple. The empty tuple corresponds to the unit type and is translated as such.

\begin{lstlisting}[language=Scala, style=short]
case class Tuple2[T0, T1](_0: MutCell[T0], _1: MutCell[T1])
\end{lstlisting}

\subsubsection{Using Data}

If a field or a variable is read, the translation introduces an access to the
\passthrough{\lstinline!value!} field of the corresponding cell. The same
happens for dereferencing. In other words, any place expression that is
evaluated in a value context is translated as reading the
\passthrough{\lstinline!value!} of the corresponding mutable cell object. This
shows how the translation unifies the access of locally available bindings and
references, that may come from elsewhere.

\paragraph{Copy vs Moving}

When place expressions are used in value contexts their data is  moved or
copied. For copyable types, the translation needs to ensure that the newly
created copy is distinct from the original object. This holds trivially for
primitive JVM types\footnote{Numeric types, \lstinline!char!s and booleans.} as
they are copied by value on the JVM as well. The same goes for shared Rust
references, where the reference is copied but still points to the same object
like on the JVM. However, tuples of copyable types, for example tuples of
numbers, are copyable as well in Rust. This is a problem for the translation,
because if it allows tuple objects to be shared by reference upon copy, the
example in \autoref{lst:tuple-sharing} is incorrect in Scala.

\begin{figure}[hbt]
\noindent\begin{minipage}[t]{.3\textwidth}
\begin{lstlisting}[language=Rust, caption={The tuple is copied on line 4.}, label={lst:tuple-share}]
let x = (123, false);

// copies `x`
let mut y = x;

y.0 = 456;
// holds:
assert!(x.0 == 123)
\end{lstlisting}
\end{minipage}\hfill
\begin{minipage}[t]{.67\textwidth}
\begin{lstlisting}[
  language=Scala,
  label=lst:tuple-sharing,
  caption={In Scala, the tuple object is shared not copied.}
]
val x =
  MutCell(Tuple2(MutCell(123), MutCell(false)))
// shares tuple object `x.value`
val y = MutCell(x.value)
// i.e. x.value == y.value
y.value._0.value = 456
// fails:
assert(x.value._0.value == 123)
\end{lstlisting}
\end{minipage}
\end{figure}


To avoid such problems, the translation introduces two operators in Scala:
\passthrough{\lstinline!move[T](t: T): T!} and
\passthrough{\lstinline!copy[T](t: T): T!}. They can be thought of as top-level
functions that perform a deep copy of their argument. An implementation for
these operators could be based on type classes for example, but implementation
details are left open here. Whenever data is used, the translation wraps the
field access of the cell's \lstinline!value! into one of the two operators,
depending on the type of the data. For example on line 4 of
\autoref{lst:tuple-sharing}, the right-hand side of the assignment becomes
\lstinline!MutCell(copy(x.value))!. This is omitted for primitive JVM types as
an optimisation.

User-defined copyable types, i.e. types for which the user derives a
\lstinline!Copy! implementation with the macro, run into the same issue as
copyable tuples. The translation currently ignores these because the current
state of our extraction implementation does not yet accept such types.
Nonetheless, the solution would be to use the \lstinline!copy! operator as well.

\subsubsection{Referencing}

A borrow in Rust creates a reference to a place and makes that reference
available as a value in the program. After translation, the place expression in
Rust is equivalent to the JVM cell object. Hence, passing around Rust references
as values can be translated as passing around cell objects in Scala. (Passing
around the cells conveniently happens by reference on the JVM.)

As an example, if a Rust reference is stored in a local variable, this creates
two nested cells in the translation. The outer cell models the local variable
and its field holds the other cell which models the Rust reference.

\noindent\begin{minipage}[t]{.48\textwidth}
\begin{lstlisting}[language=Rust, caption={Taking a mutable reference in Rust.}]
let x: T = ...;
let r = &mut x;
\end{lstlisting}
\end{minipage}\hfill
\begin{minipage}[t]{.48\textwidth}
\begin{lstlisting}[language=Scala, caption={Mentioning the cell \lstinline!x! suffices in the translation.}]
val x = MutCell[T](...)
val r = MutCell(x)
\end{lstlisting}
\end{minipage}

\subsubsection{Matching}

The same principles apply for pattern matching as for evaluating place
expressions. If a match happens on a value, the Scala match will be on the
\passthrough{\lstinline!value!} field of that cell. If the scrutinee is a
reference, then the Scala scrutinee is the cell object. The patterns are adapted
to include the mutable cell wrappers of fields. Depending on whether a resulting
binding of a match is a reference or not in Rust, the Scala pattern matches the
cell or only its field, see \autoref{lst:pattern-match}.

\begin{figure}[htb]
\noindent\begin{minipage}[t]{.48\textwidth}
\begin{lstlisting}[
  language=Rust,
  caption={In the first case, \lstinline!a! binds the value of \lstinline!x.a!
  whereas in the other cases it is \lstinline!&mut x.a!.}
]
match x {
    A { a, .. } => ...
}
match &mut x {
  A { a, .. } => ...
}
match x {
  A { mut ref a, .. } => ...
}
\end{lstlisting}
\end{minipage}\hfill
\begin{minipage}[t]{.48\textwidth}
\begin{lstlisting}[
  language=Scala,
  caption={In the first case, the binding \lstinline!a! matches \lstinline!x.value.a.value!
    of the cell, whereas in the other cases it matches the cell object, \lstinline!x.value.a!.},
  label={lst:pattern-match}
]
x.value match {
    case A(MutCell(a), _) => ...
}
x match {
    case MutRef(A(a, _)) => ...
}
x.value match {
    case A(a, _) => ...
}
\end{lstlisting}
\end{minipage}
\end{figure}


\subsubsection{Assigning}

Mutating values in both Rust and Scala is done by assigning new rvalues to
lvalues. The translation for assignments is analogous to the one for accessing
places and again unifies local variables and references. If a place expression
is assigned an rvalue, the translation assigns to the
\passthrough{\lstinline!value!} field of the cell, no matter whether the place
expression is a local variable or a mutable reference, see \autoref{lst:assign}.

\begin{figure}
\noindent\begin{minipage}[t]{.48\textwidth}
\begin{lstlisting}[language=Rust, caption={Dereferencing and assigning in Rust.}]
// `x` is of type `&mut i32`
*x = 123;
let mut y = 456;
y = 789;
\end{lstlisting}
\end{minipage}\hfill
\begin{minipage}[t]{.48\textwidth}
\begin{lstlisting}[
  language=Scala,
  label=lst:assign,
  caption={Assignments in Scala always go to the \lstinline!value! field.}
]
// `x` is of type `MutCell[Int]`
x.value = 123
val y = MutCell(456)
y.value = 789
\end{lstlisting}
\end{minipage}
\end{figure}

\subsubsection{Function parameters}

Rust functions can take parameters by reference or by value, see
\autoref{lst:byval-byref}. By-reference parameters are wrapped in a mutable cell
in the translation. Additionally, function parameters are also variables and
thereby places in Rust. To account for that and model the possible local
mutability of the parameters (when marked with \lstinline!mut!), the translation
again wraps the parameter in a mutable cell, regardless of whether the parameter
is \lstinline!mut! or not. At function call sites, all arguments are wrapped
into newly created cells.

\noindent\begin{minipage}[t]{.48\textwidth}
\begin{lstlisting}[
  language=Rust,
  caption={\lstinline!i! is only mutable inside the function.},
  label=lst:byval-byref,
]
fn f(
  mut i: i32,
  x: &i32
) { ... }
let r = 123;
f(456, &r);
\end{lstlisting}
\end{minipage}\hfill
\begin{minipage}[t]{.48\textwidth}
\begin{lstlisting}[language=Scala, caption={This results in nested cell in Scala.}]
def f(
  i: MutCell[Int],
  x: MutCell[MutCell[Int]]
) = ???
val r = MutCell(123)
f(MutCell(456), MutCell(r))
\end{lstlisting}
\end{minipage}

\subsubsection{Boxes}

Contrary to the JVM, where all objects live on the heap, Rust lets the
programmer allocate objects on the heap with boxes. However, for verification it
does not matter, whether something is on the heap or the stack as long as
general memory safety is preserved. Therefore, this distinction can be erased in
the translation and everything uses normal heap-allocated Scala objects.

\subsubsection{Memory Replace}

To allow for in-place updates of mutable references, the translation supports
the  Rust function \lstinline!std::mem::replace!. With all the machinery
introduced so far, the function can be translated as a simple feature on top of
the \lstinline!move! and \lstinline!copy! operators, as shown in
\autoref{lst:replace}. Interestingly, the translation corresponds closely to the
implementation of the function -- in unsafe Rust -- from the standard
library.\footnote{\href{}{\texttt{\color{MidnightBlue}
https://doc.rust-lang.org/src/core/mem/mod.rs.html\#815}}}

\begin{figure}
\noindent\begin{minipage}[t]{.4\textwidth}
\begin{lstlisting}[
  language=Rust,
  caption={Replacing a mutable reference's value.}
]
fn f(a: &mut A) -> A {
  std::mem::replace(
    a,
    A { a: "bar", b: 0 }
  )
}
\end{lstlisting}
\end{minipage}\hfill
\begin{minipage}[t]{.55\textwidth}
\begin{lstlisting}[
  language=Scala,
  caption={The translation of the replace function.},
  label={lst:replace}
]
def f(a: MutCell[A]): A = {
  val res = move(a.value)
  a.value = A(MutCell("bar"), MutCell(0))
  res
}
\end{lstlisting}
\end{minipage}
\end{figure}



\subsection{Correctness}
\label{correctness-claim}

\begin{quote}
Under the assumption that the original Rust program type and borrow checks, I
argue that the translation of mutable Rust references and owned types to mutable
cells in Scala is equivalent in runtime semantics, i.e.~when run the Scala
translation yields the same result as the Rust program.
\end{quote}

The Rust borrow checker ensures that there is no unsafely aliased state,
i.e.~aliased state that is mutated, in the original program. This may seem like
an unnecessarily strict assumption, but Rust's ownership system is so
fundamental to the language that it is impossible to define Rust's semantics
without borrow checking \cite{krust}. Consider \autoref{lst:move-err}, because
\lstinline!y! moves out \lstinline!x! on line 2, \lstinline!x! can never be
reused and its value is undefined, even if some code could illegally read it
again. Therefore, the borrow check assumption is necessary to reason about the
semantics of the original program.

\begin{lstlisting}[
  language=Rust,
  label=lst:move-err,
  caption={This code does not compile due to the use of \lstinline!x! on line 4, after the move.}
]
let x = A { a: "foo", b: 123 };
let mut y = x;  // `x` is moved to `y`
y.b = 456;
assert!(x.b == 123) // ERROR, can't reuse `x`
\end{lstlisting}

The translation introduces a bijection between Rust places and mutable cell
objects. As seen before, places can be variables or fields. The translation
wraps each local variable, each function parameter and each field into a mutable
cell, hence, each of these places corresponds to exactly one cell. Arrays would
differ from that but they are not yet supported.

As seen above, the translation unifies the access to values. Dereferencing and
using local variables are both done through the cell object's field. One could
say, the translation lifts everything into references; even local variables are
accessed through the single gateway to the value that is the cell object.
Furthermore, the cell's field is the only location where program data can
reside. This corresponds to Rust's principle that data is always owned uniquely
(by a cell) and otherwise referenced (through a cell).

More formally, I examine in the following the two ways in which data can be
accessed in Rust, by value or by reference, to argue that the translation
results in correct Scala runtime semantics.

\subsubsection{By value}
\label{sec:correctness-by-value}

If data is used by value, it grants ownership to the new binding. This holds for
both moveable and copyable types. Only for the latter, the original binding and
data stay intact, whereas for moveable types, the original owner is invalidated
and the data de-initialised. In both cases, changes will not appear on the
original binding if the used data is later mutated. Therefore, the translation
is correct in inserting the \lstinline!move! or \lstinline!copy! operator --
which for this argument can be taken as deep copies -- around by-value reads of
a cell's data.

Performing a deep copy for moveable types is correct because the original
binding can never be reused. Thus, one can think of moving as performing a deep
copy of the original object, assigning it to a new binding and then destroying
the original object. Importantly, the changes made to the new, moved object
never propagate back to the original.

For copyable types, the translation distinguishes between aggregate types and
primitive types. Primitive JVM types are handled like in Rust and need no
further examination. Aggregate types like tuples of numbers are translated to
Scala objects with mutable cells as fields. Therefore, I need to show that the
deep object copy the \lstinline!copy! operator performs is equivalent to the
implicit bitwise copy of the struct in Rust. Indeed, a bitwise copy for an
aggregate type is trivially the same as a deep copy, as long as all contained
types are present by value. That means, all contained data is embedded in the
bits of the aggregate structure and is simply copied deeply (bitwise) as well.
It gets more complicated for shared references (mutable references are not
copyable as they need to be unique). For a shared reference, Rust simply copies
the reference, i.e.~the pointer value, but not the referenced object. Still,
that operation is equivalent to a deep copy because the reference is shared.
Thus, by the borrow check assumption, as long as any shared reference to an
object exists, that object cannot be mutated. In other words, under that
assumption a shared reference will never see its referenced value change.
Therefore, it is equivalent to copy that object deeply when needed, as it
doesn't change. The sharing can be viewed simply as an optimisation in Rust.

The above holds for tuples, arrays and user-defined \lstinline!Copy! types as
long as the \lstinline!copy! operator supports them in Scala. Other copyable
types like function pointers are not supported by the translation nor the
extraction.

\subsubsection{By reference}

The counterpart of using data and transferring its ownership is taking
references to places. The translation handles that by using the place's cell
object which is defined and unique, as shown earlier. Thanks to the cells, it
becomes possible to reference places on the JVM; it suffices to mention, i.e.~to
use, the correct cell object whenever a reference is needed. Cells are JVM
objects, hence, they can simply be used in a by-value manner and reference
handling is done by the JVM. In particular, when a cell is used in multiple
locations it will not be cloned but all the locations will point to the same
cell object. This exactly corresponds to the semantics of shared Rust references
that are model by the cells and is thus correct.

For mutable references, the mechanism is entirely the same. Any changes made on
a cell's value will be visible to all readers, because they share the cell
object. This sounds dangerous but the borrow check assumption guarantees that no
illegal sharing or aliasing can occur. For example, it is guaranteed that all
shared references to an object end their lifetime \emph{before} the object is
mutated. Thus, in the translation, a shared reference, in the form of a
reference to a cell object, might still exist somewhere on the JVM when the cell
is mutated and before the garbage collector passes, but it will never be read,
thanks to the borrow checking. Therefore, the translation is correct for mutable
references as well.


\section{Translation for Stainless}

The translation presented so far results in correct Scala runtime semantics, but
the motivation was to ultimately use the translation to verify the Scala program
with Stainless. Intuitively, the translated program can just be submitted to
Stainless to get a verification result and deduce verification properties for
the Rust program. That approach is correct because both programs exhibit the
same runtime semantics. However, Stainless has some limitations concerning its
support for mutability. Therefore, I adapt the general translation step-by-step
to yield Scala programs that are accepted by Stainless.

This section presents the changes made to the translation to be compatible with
Stainless at a theoretical level. Like before, I will argue  that all the
deviations from the general translation uphold the overall correctness. For more
low-level implementation details and an overview of the entire system please
refer to \autoref{implementation}.



\subsection{Avoiding Aliasing}
\label{sec:aliasing-restrictions}

The main difference between full Scala and the imperative fragment supported by
Stainless \cite[section ``Imperative'']{stainless-doc} is that Stainless imposes
aliasing restrictions on the program. The restrictions permit a simple but
strict aliasing invariant:

\begin{quote}
{[}For{]} each object in the system, each path of pointers reaches a
distinct area of the heap. \cite[p.~59]{regb}
\end{quote}

Concretely, the following restrictions apply to mutable types, i.e. types that
may contain \lstinline!var! fields or type parameters marked with
\lstinline!@mutable!:

\begin{itemize}
\tightlist
\item For a function call, no mutable arguments of the call can share
any part of their memory. E.g. it is not allowed to give an object as one
argument of a function call and a field of the same object as another argument.

\item
  More generally, ``it is forbidden to assign a mutable variable to a
  new identifier'',
\item
  and lastly, ``a function cannot return an object that shares memory
  locations with one of its parameters'' \cite[p.~59]{regb}.
\end{itemize}

As the translation uses mutable cells everywhere, virtually all resulting Scala
types are mutable and fall under these restrictions. On the other hand, the
motivation of Stainless's aliasing restrictions is to make its imperative code
elimination phase work \cite{regb}. That transformation relies on the fact, that
there is a single path to a mutable object. However, the restrictions are quite
severe and there are plenty of cases where an alias is safe but is not allowed
by these rules.\footnote{For example, shared references, i.e.~aliases, of a
mutable object but the object is never changed.}

In Rust, the borrow checker ensures a very similar property: there is a
\emph{unique} mutable reference to a place or otherwise it is safe to have
multiple immutable references. The borrow check assumption guarantees that the
program does not have illegal aliasing of mutable state, but the translation
needs to convince Stainless of that fact. It does so with the \lstinline!move!
and \lstinline!copy! operators. The translation already introduces
\lstinline!move! and \lstinline!copy! when data is used by value. To work around
Stainless's aliasing restrictions, it additionally introduces \lstinline!copy!
for function arguments that are shared references and lastly, it adds
\lstinline!move! or \lstinline!copy! to the return values of functions and inner
blocks, unless the values contain mutable references. The correctness of that
approach follows from the same argument as the correctness of using
\lstinline!move! and \lstinline!copy! for using data by value (see
\autoref{sec:correctness-by-value}).

To summarise, all types are \lstinline!move!'d or \lstinline!copy!'d when they
could create an alias except for types that are or contain mutable references.
The borrow checker has made sure that the latter are not aliased. Hence, the
resulting program contains no aliasing of mutable types but still preserves
correct runtime semantics.

\subsubsection{\texttt{move} is the identity}

This subsection shows that the \lstinline!move! operator -- though very useful
to work with Stainless -- is not necessary to the semantic correctness of the
translation and may be omitted by implementations that do not need to comply
with Stainless's aliasing restrictions.

\begin{quote}
Under the assumption that the original Rust program type and borrow checks, the
\lstinline!move! operator is semantically equivalent to the identity and simply
serves as a way to make verification work with a simple aliasing analysis.
\end{quote}

As seen before, one can think of moving as performing a deep copy and
invalidating the original. The copying during a move avoids creating an alias to
the source cell. But, the borrow checker guarantees that a moved out value is
never used again, hence the original will never be read again and its value does
not matter for the correctness of the program, it might as well change.

It is easier to see the argument in the example of \autoref{lst:move-err}. On
line 2, the struct is moved out of \lstinline!x! into \lstinline!y!. That is,
without the \lstinline!move! operator, an alias of \lstinline!x!'s cell would be
created, but the binding of \lstinline!x! will never be used again. Hence,
\lstinline!x! can be changed without problems and the inserted \lstinline!move!
only serves to communicate this fact to the simple aliasing analysis of
Stainless. The same holds for function arguments and return values. Thus, in
order to only achieve correct Scala runtime semantics, the \lstinline!move! can
be omitted because it is equivalent to the identity.

\subsubsection{\texorpdfstring{\texttt{copy} is the functional
identity}{copy is the functional identity}}

The second claim only concerns completely functional programs, that is, programs
that do not mutate any data.

\begin{quote}
Under the assumption that the original Rust program type and borrow checks, and
in the absence of \emph{all mutation}, the \lstinline!copy! operator is
semantically equivalent to the identity and simply serves as a way to make
verification work with a simple aliasing analysis.
\end{quote}

Consider \autoref{lst:tuple-sharing}, the introduced aliasing of the tuple
object on line 4 is fixed by adding the \lstinline!copy! operator, but the
aliasing only poses a problem because the tuple is later modified. If the tuple
wasn't modified, sharing would be safe and the program correct.

More generally, in the absence of mutation, there are only shared references and
immutable by-value bindings. If data needs to be copied according to Rust
semantics but is immutable, then it is safe to share the cell object in the
translation because the data cannot be changed at any time. The same holds for
shared references. Remember that a shared reference can never see its referenced
data change, therefore it is safe to deeply copy the referenced object. But
again, if that data cannot change at all, one may as well share the objects.
This shows that the \lstinline!copy! operator is equivalent to the identity in
programs without mutation.

\subsubsection{Cloning is copying}
\label{sec:clone-is-copy}

Unlike the built-in implicit copy that occurs for copyable Rust types, a clone
always happens explicitly by calling \lstinline!clone()!. Programmers can define
their own implementation of the \lstinline!Clone! trait or they can let the
\lstinline!derive! macro generate one.

\begin{quote}
Under the assumption that all \lstinline!Clone! implementations in a program are
derived by the macro, \lstinline!clone! is equivalent to a deep copy and can be
translated with the \lstinline!copy! operator.
\end{quote}

Derived \lstinline!Clone! implementations for struct types recursively call
\lstinline!clone! on all the fields and return a new struct with the cloned
fields. For copyable types, especially primitive types, the clone is trivially a
bitwise copy like for
\lstinline!Copy!.\footnote{\url{https://doc.rust-lang.org/std/clone/trait.Clone.html}}
For shared references, \autoref{sec:correctness-by-value} shows that copying is
equivalent to a deep copy of the referenced object.

By a simple inductive argument, if the \lstinline!Clone! implementations of all
the fields are equivalent to deep copies, then so is the struct's
\lstinline!Clone! implementation. Hence, as long as all component types
recursively are copyable types or have derived \lstinline!Clone!
implementations, the top-level implementation is equivalent to a deep copy. The
translation can therefore check that assumption and if it holds, translate calls
to \lstinline!clone! with the \lstinline!copy! operator in Scala, which is
equivalent to the identity in completely functional programs or performs a deep
copy otherwise.



\subsection{Optimisations}
\label{optimisations}

Even if the translation presented here serves to add mutability features to
Rust-Stainless, it should preserve the well-functioning parts of the frontend
that deal with immutable Rust. To that end, the translation that is implemented
in software contains some optimisations that increase the usability of
Rust-Stainless for functional Rust code. The following subsection presents these
optimisations and argues that they uphold the overall correctness.


\subsubsection{Immutable Bindings and References}

The most important optimisation is that immutable bindings (local variables,
function parameters) are not wrapped in mutable cells. Moreover, shared
references are erased, like boxes, and their values  are used directly. This
optimisation is possible thanks to excellent support of the Rust compiler which
provides type information for every expression. With that, it is possible to
distinguish immutable values and shared references at all times from possibly
mutable ones that are still wrapped in cells.

To convince the reader of the correctness of this optimisation, first note that
only mutable data can be borrowed mutably. In other words, only for mutable data
do we need the reference sharing in the translation, to propagate changes back
to the original object. Data declared without \lstinline!mut! is immutable and
cannot change in the first place. Secondly, it is safe to directly use immutable
values instead of references, even if the values get copied at some points,
which again follows  from the argument that one may deeply copy shared
references (cf. \autoref{sec:correctness-by-value}). Therefore, it is correct to
erase shared references and use their referred values directly.

\subsubsection{Preventing Aliasing}

Previous sections showed how \lstinline!move! and \lstinline!copy! are used to
emit code that complies with Stainless's aliasing restrictions. They also argued
that \lstinline!copy! \emph{must} perform a semantic deep copy, while
\lstinline!move! \emph{may} do a deep copy but might also be the identity. Our
Stainless-backed implementation uses a newly added Stainless primitive to
emulate both operators: \lstinline!freshCopy[T](t: T): T! semantically performs
a deep copy of its argument. It does not matter that many deep copies are
introduced in the program in that way. The program is never run with these deep
copies, they only serve for verification.

The implementation decides for each location where data is used by value whether
to insert a \lstinline!freshCopy! or not based on the type of the data. Types
that are or contain mutable references, like \lstinline!Option<&mut T>!, are not
\lstinline!freshCopy!'d and the copy is also omitted for primitive JVM types.
Additionally, the translation avoids \lstinline!freshCopy! around some control
structures for which it is clear by design that a \lstinline!freshCopy! will be
inserted inside the structure. For example in a \lstinline!match! expression,
the resulting value of each arm will be copied and therefore the entire match
expression doesn't need to be wrapped in an additional \lstinline!freshCopy!.

The same procedure is applied for cloning. Conveniently, the Rust compiler does
not derive \lstinline!Clone! implementations for types that are or contain
mutable references.\footnote{\href{}{\texttt{\color{MidnightBlue}
https://doc.rust-lang.org/std/clone/trait.Clone.html\#impl-Clone-121}}}
Therefore, it is sufficient to check that all \lstinline!Clone! implementations
are derived (cf. \autoref{sec:clone-is-copy}) to ensure that all types for which
\lstinline!clone! is translated will be \lstinline!freshCopy!'d.

To summarise, the implementation deeply copies some of the moves and all the
copies that happen in Rust to prevent aliasing in the resulting code. But, it
never copies mutable references because for those, the Rust compiler has made
sure that there are no aliases. Lastly, the same happens for cloning because, by
the check, only clone operations are extracted for which these semantics are
correct.


\subsection{Limitations}
\label{trans-limitations}

So far, mutable cells together with the \lstinline!freshCopy! primitive solved
all cases of aliasing restriction problems. However, there are patterns where
that solution is not applicable and these constitute the theoretical limits of
this translation with the current aliasing restrictions of Stainless -- the
runtime equivalence in full Scala is still valid.

\begin{figure}
\noindent\begin{minipage}[t]{.35\textwidth}
\begin{lstlisting}[
  language=Rust,
  caption={A mutable reference in a mutable variable.}
]
let mut a = 123;
let mut b = 456;
let mut x = &mut a;

x = &mut b;
\end{lstlisting}
\end{minipage}\hfill
\begin{minipage}[t]{.6\textwidth}
\begin{lstlisting}[
  language=Scala,
  label=lst:nested-cell,
  caption={Translation with nested mutable cells.}
]
val a: MutCell[Int] = MutCell[Int](123)
val b: MutCell[Int] = MutCell[Int](456)
val x: MutCell[MutCell[Int]] =
  MutCell[MutCell[Int]](a)
x.value = MutCell[Int](b)
\end{lstlisting}
\end{minipage}
\end{figure}

The unsupported patterns occur when mutable references are used in locations
where they cannot be \lstinline!freshCopy!'d to keep the semantics correct. For
example, a mutable binding of a mutable reference: \passthrough{\lstinline!mut
x: &mut T!}. The variable \lstinline!x! can be dereferenced and assigned
\lstinline!*x = ...!, which changes the referred object, but it can also be
reassigned with another mutable reference \lstinline!x = &mut ...!. To correctly
model both of these cases, the translation gives the type
\lstinline!MutCell[MutCell[T]]! to \lstinline!x!, like in
\autoref{lst:nested-cell}. Note that the example runs correctly in Scala.
Unfortunately, the same cannot be said about Stainless. The assignment on line 5
creates an alias of \lstinline!b! in \lstinline!x.value!, which Stainless does
not permit. This can occur for local variables as in the example but also for
\lstinline!mut! function parameters of mutable reference type.

\begin{lstlisting}[
  float=htb,
  language=Rust,
  label=lst:recursive-get-mut,
  caption={A recursive method returning a mutable reference.}
]
impl<V> List<(u128, V)> {
  pub fn get_mut(&mut self, key: &u128) -> Option<&mut V> {
    match self {
      List::Nil => None,
      List::Cons(head, _) if head.0 == *key => Some(&mut head.1),
      List::Cons(_, tail) => tail.get_mut(key),
    }
  }
}
\end{lstlisting}

The other case of mutable references, i.e. mutable cells, that cannot be
\lstinline!freshCopy!'d occurs in function return values. In the running example
of \autoref{running-example}, \lstinline!get_mut_by_id! returns a mutable
reference. This is correctly translated and also verified. However, Stainless
does not allow the same for recursive functions, because of the third aliasing
rule in \autoref{sec:aliasing-restrictions}. Thus, the same \lstinline!get_mut!
method for a list which is recursive, shown in \autoref{lst:recursive-get-mut},
is rejected by Stainless. \newpage

\hfill \break \noindent This chapter presented a translation for mutability from
Rust to Scala and refined it in several steps to make it useable with the
current imperative phase of Stainless. The general translation is quite powerful
and results in equivalent runtime semantics in Scala for mutability features of
Rust. For each version of the translation, I argued that the runtime equivalence
is upheld. The last section discussed the theoretical limitation of the
approach. Further limitations more associated with the implementation of this
translation are discussed in \autoref{impl-limitations}. More generally, the
next chapter sheds light on the engineering of Rust-Stainless and other new
features than mutability.

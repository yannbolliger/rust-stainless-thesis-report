This chapter situates the work on the translation in the context of existing
research on linearly typed languages and similar approaches. It takes a closer
look at some formalisations of Rust's semantics and finally compares our tool to
other projects that work on verification of Rust code.


\section{Background Topics}

Rust's type and ownership system has been heavily inspired by decades of
research in the programming language community. The main topics to consider are
linear and unique types, ownership, as well as region-based memory management.

Linearity was introduced by \citet{girard} and \citet{Wadler90lineartypes}.
Linear types must be used \emph{exactly once}, they cannot be duplicated nor
discarded \cite{use-once}. This presents the disadvantage that values to be
reused have to be threaded through the program, e.g.~a function that reads from
an array also needs to return the array after the read to make it useable again.
Unique types as described by \citet{alias-free-pointers} on the other hand
guarantee that they are the only reference to a certain value, i.e.~the absence
of aliases. Lastly, \citet{ownership-types} introduced ownership types that
follow the same goal, the absence of inadvertent and even dangerous aliasing of
(heap-allocated) objects.

Many early approaches to linear and unique types originated in functional
languages and some challenges arose when they were applied to object-oriented
languages that may assign to values or clone objects. Shallowly copying an
object in a language with owned data is problematic because it might copy unique
pointers, creating illegal aliases. Deep copying would solve the problem but
there is a more efficient approach called \emph{sheep cloning}. First described
by \citet{dynamic-alias-protection}, Rust implements a version close to
\cite{sheep-cloning}: owned data, that is, data pointed at by unique pointers,
is deeply copied, while shared data, i.e.~immutable references, can be safely
aliased.

To make unique types work with assignments, values need to be moved or destroyed
after a read. This notion of \emph{destructive reads} was introduced by
\citet{islands-alias-protection}, with the goal of protecting from aliasing in
object-oriented languages. Rust implements such reads for its move semantics by
making the borrow checker flow-sensitive, similar to the proposition of
\citet{alias-burying}. The flow-sensitive borrow checker can also be seen as
substructural typing \cite{oxide}.

Other, less practically used languages with substructural typing or
similar ideas are:

\begin{itemize}
\item \textbf{Mezzo}, an ML-language by \citet{mezzo}, controls aliasing and
ownership by offering \emph{duplicable types} that are immutable and can be
copied, and \emph{mutable types} that are linear. This is similar to Rust's
copyable and moveable types. Mezzo also uses a flow-sensitive type checker to
enforce its discipline but it does that with permissions embedded in the type
system rather than lifetimes.

\item \textbf{Linear Haskell} by \citet{linear-haskell} tries to bring linear
types to Haskell in a backwards-compatible way, i.e.~existing code compiles
alongside linear types. The key difference of this approach is that linearity is
attached to functions instead of having linear and non-linear types like Mezzo
or Rust. A linear function is restricted in the way it uses its arguments: when
its result is consumed exactly once, then its argument has to be consumed
exactly once.

\item \textbf{Alms} is an attempt to popularise \emph{affine types}, e.g.~types
that can be used \emph{at most once}, by \citet{alms}. The language is similar
to OCaml and its primary goal is to create resource-aware abstractions, for
solving problems like race conditions. In an Alms module, the programmer can
mark types as affine, contrary to Rust where everything is moveable,
i.e.~affine, unless otherwise specified (by implementing \lstinline!Copy!).

\item \textbf{Cyclone} was intended as a save alternative to C and is among the
languages that inspired Rust the most. In particular, \citet{cyclone-region}
designed Cyclone with memory management based on regions. That includes ideas
like region subtyping and deallocation by region. Unlike Rust, the programmer
specifies whether an object should be on the heap, on the stack or in a dynamic
region. In Rust, the lifetime mechanism is applied to all references and memory
locations, ownership is used to trigger deallocation.
\end{itemize}


\section{Rust Formalisations}

Even if Rust is a rather young language, there have already been numerous
attempts at formalising its semantics of ownership, borrowing and reference
lifetimes. The following list mentions three important and recent works. Other
projects are Patina \cite{patina}, Rusty Types \cite{rusty-types}, and KRust
\cite{krust}.

\begin{itemize}
\item \textbf{RustBelt} by \citet{rustbelt} is important because it bridges the
gap of most other works: it proves the safety of implementations in \emph{unsafe
Rust} -- ``securing the foundations of Rust''. The authors develop an automated
way of proving correctness for a subset of the language and apply it to
libraries that internally use unsafe features. For example, they prove that a
program using the abstractions of \passthrough{\lstinline!std::cell!} runs
safely if it type checks. In contrast to the following works and Rust-Stainless
however, RustBelt uses a continuation-passing style language,
\(\lambda_{Rust}\), that is closer to the MIR than Rust itself.

\item \textbf{Oxide} is a recent formalisation by some of Rust's maintainers,
\citet{oxide}. The goal of the project is to provide formal semantics for a
language close to surface Rust called Oxide. The authors present the first
syntactic proof of type safety for the borrow checking of (the considered subset
of) Rust and implement a type-checker for that language. The interpretation of
lifetimes in Oxide is already compatible with the future version of Rust's
borrow checker called \emph{Polonius} \cite{polonius}.

\item \textbf{Lightweight Formalism for Rust (FR)} was developed in parallel to
Oxide by \citet{fr}. Like Oxide, it draws strong inspiration from Featherweight
Java (FJ) \cite{fj}, hence the name FR. The paper presents a calculus that
models source-level Rust with its salient features like borrowing, reference
lifetimes and move versus copy semantics. The author proves the soundness of the
calculus but unlike Oxide, they implement the system in Java rather than in
Rust. Another difference to Oxide is that FR models boxes, i.e. heap allocation.
Oxide on the other hand is closer to the future version of the borrow checker
and covers a larger subset of Rust.
\end{itemize}

The two latter works are of particular importance to this project because these
formalisations could be used to formally prove the correctness of the
translation from \autoref{translation}.

\section{Verification}

Stainless in its current form, with the \emph{Inox} solver \cite{inox}, as
developed by \citet*{stainless} is the successor of Leon \cite{leon}. The two
systems come out of a long series of works, exploring the possibilities of
verifying functional programs, mainly in Scala, with SMT solvers \cite{smt},
like the early work by \citet{smrp}. Stainless distinguishes itself from many
other verifiers in that it is \emph{counter-example complete}, that is, it
produces and minimises counter-examples for all failed verification conditions.

Of particular interest to this project is the imperative phase of Stainless that
was introduced by \citet{regb} and acts as main backend to the translation
presented in this thesis. The new imperative phase by \citet{new-imperative} is
likely to overcome the current limitations of imperative code in Stainless and
may be the ideal target for Rust-Stainless in the future.

\subsubsection{Rust Verification}

Due to its clear design and promising safety guarantees, Rust is predestined as
implementation language for critical systems. With that arises the need to prove
stronger properties on Rust programs like functional correctness. This is the
major motivation behind this project, but likewise, numerous other projects try
to build verifiers for Rust. The wide interest in the topic can be seen from the
sheer number of projects in the following list. The selection here focusses on
the projects that are closer to Rust-Stainless.\footnote{The list of verifiers
largely stems from
\url{https://alastairreid.github.io/rust-verification-tools/}.}

\begin{itemize}

\item \textbf{CRUST} \cite{crust} tries to prove memory safety of Rust code
using unsafe features, in the same manner as RustBelt. The approach is to
translate Rust to C, generate test sequences of function calls, and then verify
the code with the CBMC bounded model checker for C~\cite{cbmc}.

\item \textbf{RustHorn} \cite{rusthorn} translates Rust into \emph{constrained
Horn clauses} that can be solved by an appropriate solver. The authors prove
correctness of their translation for a formalised subset of Rust including
mutable references, inspired by RustBelt \cite{rustbelt}. Contrary to
Rust-Stainless, RustHorn operates on the MIR. The tool mainly proves the absence
of runtime errors and has no features to  express and verify specifications or
higher-level properties.

\item \textbf{Seer} \cite{seer} stands for \emph{symbolic execution engine for
Rust}. The engine takes the MIR and executes it symbolically with Z3 \cite{z3}
as a solver backend, similar to Stainless. However, as an execution engine, the
tool tries to find executions that produce errors but cannot prove higher-level
properties.

\item \textbf{SMACK} is a verification toolchain that translates LLVM-IR to
Boogie \cite{boogie} and  has been extended to support Rust \cite{smack}. Like
RustHorn, it is able to prove the absence of runtime errors and correctness of
assertions but no higher-level properties like laws in Rust-Stainless.

\item \textbf{MIRAI} \cite{mirai} is an abstract interpreter for MIR, developed
at Facebook. It supports very similar pre- and postconditions like
Rust-Stainless that are standardised by the \lstinline!contracts!
crate.\footnote{\url{https://docs.rs/contracts/latest/contracts/}} The
repository also contains contracts for many modules of the standard library.
Such contracts could be a solution to Rust-Stainless's one-crate-limitation
(cf. \autoref{impl-limitations}).  Contrary to Stainless, MIRAI may produce
false negatives and  is unable to generate counter-examples for failed
verification conditions.

\item \textbf{Electrolysis} \cite{electrolysis} is probably the most similar
work to Rust-Stainless. It translates Rust to the functional language Lean which
is used as an interactive theorem prover \cite{lean}. Like Rust-Stainless,
Electrolysis only works on the safe subset of Rust and one of the core topics of
the project is the translation of mutable references. The author chooses
functional lenses as representation of mutable references in Lean. This has the
limitation that the translation needs to keep track of the provenance of mutable
references. Unlike Rust-Stainless, this project operates on the MIR.

\item \textbf{Creusot} \cite{creusot} is similar to Electrolysis, it also
translates Rust to a verification language, WhyML \cite{why3}. Like MIRAI and
Rust-Stainless, Creusot lets  the programmer state specifications as pre- and
postconditions. However, it uses a special language for the contracts, called
\emph{Pearlite}, that is developed with the project. Creusot also works with the
MIR.

\item \textbf{Prusti} \cite{prusti} is probably the most useable tool currently,
as it is the only one to offer a VSCode extension that verifies properties
while the programmer is writing code. Prusti translates MIR to Viper
\cite{viper}, a verification framework with frontends for various mainstream
languages. Like Rust-Stainless, pre- and postconditions are expressed in Rust
itself. Prusti also has the ability to attach contracts to crate-external items,
similar to MIRAI, solving the one-crate-limitation (cf.
\autoref{impl-limitations}). On the other hand, Prusti does not yet offer trait
contract verification like Rust-Stainless with laws.

\end{itemize}

\hfill \break \noindent  From this extensive enumeration, it is clear that
Rust-Stainless is currently the only tool that operates on an IR above the MIR
-- the THIR. The trade-offs involved in that decision are discussed in
\autoref{mir-thir}. Rust-Stainless could draw an advantage from that decision in
the future and provide counter-example generation for Rust. That and other
promising paths for the future of the tool are explored in the next and last
chapter.

\section{The Rust Language}

Rust \cite{rust2, rust1} is a recent systems programming language, initally
developed at Mozilla with the goal to replace C++ as application programming
language. The most distinguishing feature of Rust is its new approach to memory
management. Traditional languages tend to either manage memory with a
\emph{garbage collector} or require the programmer to manually allocate and,
more importantly, deallocate memory. Rust is able to deallocate memory itself
without a garbage collector by using concepts of \emph{ownership} and
\emph{reference lifetimes}. In other words, Rust provides memory management at
compile time which results in cost-free memory management at runtime.
Additionally, Rust guarantees memory safety properties like  the absence of
\emph{dangling pointers}, i.e. it is impossible to dereference a freed part of
memory.

To achieve this, Rust introduces a flow-sensitive type checker, the \emph{borrow
checker}, which is a concrete implementation of existing ideas that have been
widely studied in many years of research. The two main concepts are ownership
\cite{ownership-types} and \emph{borrowing} that build on the research topics of
\emph{uniqueness} \cite{external-uniqueness, alias-burying} and \emph{linearity}
\cite{Wadler90lineartypes, once-upon-a-type, linear-haskell}, and reference
lifetimes that are inspired by work on region-based memory management
\cite{cyclone-region}.


\subsection{Syntax Overview}

As a start, the \autoref{lst:first-ex} shows a simple function in Rust that
calculates and returns the square of the given integer. The language is
statically typed, therefore the parameter \lstinline!x! has its type annotated,
\lstinline!i32!, a signed 32-bit integer. Other integer types can be unsigned
(\lstinline!u32!) or of other bitwidths (\lstinline!u128!). The return type of
function is indicated after the arrow, if omitted the return type defaults to
\lstinline!-> ()!, the unit type. Note also, that the expression on the last
line of the function, or any brace-delimited block is implicitly returned.
Alternatively, one may also employ the explicit return statement to return a
value from any point in a function, like in \autoref{lst:let}. Statements
are delimited with semi-colons and produce the unit type.

\begin{lstlisting}[language=Rust, caption={A simple Rust function.}, label=lst:first-ex]
fn square(x: i32) -> i32 {
  x * x
}
\end{lstlisting}

\subsubsection{Variables}

To bind values, Rust offers the \lstinline!let! statement, which creates a new
variable and binds the assigned value to it. This is the case for both
\lstinline!x, y! in \autoref{lst:let}. Function parameters are also
variables~\cite{rustref}. By default, variables are
immutable, that is the compiler rejects any attempt to reassign to an
initialised variable. On the other hand, it accepts to redeclare and thus
\emph{shadow} a binding, which is what happens on line 7 of \autoref{lst:let}.

\begin{lstlisting}[language=Rust, caption={A Rust function doing some arithmetics.}, label=lst:let]
fn f(z: i32) -> i32 {
  let x = z * 10;
  if x > 100 {
    return z;
  }
  let y = z * 100;
  let z = 3;
  x + y + z
}
\end{lstlisting}

Something to note for let-bindings is that in most cases, the type of variable
does not have to be declared like for function parameters but rather the
compiler is able to infer it. In \autoref{lst:let}, all the types are inferred
as the standard integer type \lstinline!i32!. This even works when a binding is
shadowed with a different type from the earlier declaration.

\subsubsection{Mutability}

As an imperative language Rust nonetheless offers mutable bindings, but each
variable has to be explicitly declared as mutable by annotating it with the
\lstinline!mut! keyword. In general, Rust is always very explicit about the
distinction between mutable and immutable variables and references.
\autoref{lst:mut-ex} has both the function parameter as well as the local
variable \lstinline!b! mutable.

\begin{lstlisting}[language=Rust, caption={Mutable variable bindings.}, label=lst:mut-ex]
fn g(mut a: i32) -> i32 {
  let mut b = 10;
  a += b;
  b = 100;
  a * b
}
\end{lstlisting}

\subsubsection{Algebraic Data Types (ADT)}

Taking strong inspiration from C, user-defined data types in Rust are called
\lstinline!struct!s and \lstinline!enum!s. Structs are product types and come in
three forms: without fields like \lstinline!OnlyAMarker! in
\autoref{lst:struct-ex}, so called \emph{unit structs}, the tuple form with
numerically indexed fields, e.g. \lstinline!a.0! and the C-like structs with
field names, \lstinline!b.field!. Enumerations are sum types and can have
members that are again like structs in either unit, tuple or C-like form.

\begin{lstlisting}[language=Rust, caption={ADTs an instances thereof.}, label=lst:struct-ex]
struct OnlyAMarker;
struct A(i32, bool);
struct B {
  field: u8
}
enum C {
  SecondMarker,
  TupleLike(i128),
  StructLike {
    a: bool
  }
}
let a = A(0, true);
let mut b = B { field: 2 };
let c = C::TupleLike(123);
\end{lstlisting}

Note that fields are not considered variables in Rust, rather they are a part of
the struct's variable~\cite{rustref}. Hence, fields inherit the mutability of
their struct's variable, e.g. the only mutable field in \autoref{lst:struct-ex}
is \lstinline!b.field! because \lstinline!b! is mutable. That means mutability
is not part of a fields type but depends on the binding of each instance.

Finally, Rust also has anonymous tuples. These behave exactly like tuple structs
but don't need to be declared upfront. The programmer can simply instatiate
tuples of any non-negative number of types, like
\lstinline!let t: (i32, bool) = (123, true)!.


\iffalse
  - pattern match
  - impls
  - traits
  - generics
\fi


\subsection{Owning and Referencing Data}

\subsubsection{Value Categories}

Rust, like C but unlike Java or Scala, exposes memory management to the
programmer and makes it possible to reference data that is locally allocated on
the stack. Therefore, the programmer and compiler need  to distinguish between
different \emph{value categories} of an expression:

\begin{itemize}
\tightlist
\item
  \emph{lvalues} are expressions that designate memory locations, for
  example variables, array elements etc. They are objects that have a
  storage address. The term originates from the fact that lvalues are on
  the left-hand side of an assignment.
\item
  \emph{rvalues} are expressions or temporary values that do not
  persist. They are on the right-hand side of an assignment \cite{wiki:lvalues}.
\end{itemize}

These two terms were replaced in Rust by \emph{place and value expressions}.
Value expressions represent actual values and are defined by exclusion from
place expressions. The latter evaluate to a \emph{place} which is essentially a
memory location~\cite{rustref}.

\begin{quote}
{[}Place{]} expressions are paths which refer to local variables,
{[}\ldots{]} dereferences (\passthrough{\lstinline!*expr!}), array
indexing expressions (\passthrough{\lstinline!expr[expr]!}), field
references (\passthrough{\lstinline!expr.f!}) {[}\ldots{]}. \\
\cite[section "Expressions"]{rustref}
\end{quote}

Place expressions occur in \emph{place expression contexts} like on the
left-hand side of let-bindings, as borrow operands, as field expression
operands or as pattern match scrutinees. Otherwise, if a place
expression is evaluated in a \emph{value context}, e.g.~on the
right-hand side of a binding, it evaluates to what is stored at the
place it evaluates to. We say, its data is \emph{used}.

\subsubsection{Ownership}

As said previously, the distinguishing feature of Rust is its ownership system.
That system is responsible for

\begin{quote}
{[enforcing]} an ownership invariant where a
variable is said to "own" the value it contains such that no two variables can
own the same value \cite[page 5]{lightweight-formalism}.
\end{quote}

In other words, data is owned by exactly one binding and if some data is used in
an assignment or passed to a function, it will be \emph{moved out} of that
binding, i.e. the ownership of the data is transferred to the new binding or the
function. The old binding is deinitialised and can never be reused, we say it
behaves linearly \cite{oxide}. One can also see moving out as \emph{destructive
read} \cite{islands-alias-protection}. In \autoref{lst:move}, data is moved out
of \passthrough{\lstinline!a!} and its ownership is transferred to
\passthrough{\lstinline!b!}.

\begin{lstlisting}[
  language=Rust,
  label=lst:move,
  caption={A struct type with move semantics.}
]
struct A { a: i32 };
let a = A { a: 123 };
let b = a;
// `a` can never be used from here
\end{lstlisting}

From \autoref{lst:move} it is clear, why Rust's borrow checker needs to be
flow-sensitive. The move semantics render the variable \lstinline!a! dead before
it formally goes out of its scope. For example, a subsequent statement like
\lstinline!let c: A = a;! would type check because the type checker is
flow-insensitive, however the borrow checker needs to take flow into account.
This is also applies to single fields of structs, e.g. a field can be moved
out of a struct and become inaccessible while another field is still available.
This is called a \emph{partial move}. The struct itself also becomes
inaccessible as soon as one of its fields is moved.

\paragraph{Move vs. Copy}

One departure from these \emph{use once} \citep{use-once} variables are copyable
types. Types that implement the \passthrough{\lstinline!Copy!} trait in Rust get
copy semantics rather than move semantics. The compiler automatically implements
this for primitive types\footnote{Primitive types in Rust are \lstinline!bool!,
\lstinline!char!, \lstinline!str!, numeric types and the type for panics
\texttt{!}~\cite{rustref}} (cf. \autoref{lst:copy}) but also for shared
references and tuples of copyable types \cite[section "Special types and
traits"]{rustref}. When data of these types is used, it's implicitly copied and
the original binding and data stay valid and independent. In contrast to shared
references that can be copied, mutable references have move semantics because
they need to stay unique, as will become clear shortly.

\begin{lstlisting}[
  language=Rust,
  label=lst:copy,
  caption={Copy semantics of the \lstinline!i32! type.}
]
let a = 123;
let mut b = a;
b += 1;
// `a` can still be used and is == 123
\end{lstlisting}

For types that are not copyable, we are still able to explicitly create
copies of objects by implementing the \passthrough{\lstinline!Clone!}
trait and calling the \passthrough{\lstinline!Clone::clone!} method.

\paragraph{Heap Allocation}

For now, all the variables were allocated on the stack and hence, tied to the
scope of their surrounding function. Heap allocation in Rust is achieved with
the \lstinline!Box<T>! type. Everything, that is put into a box, is directly
allocated on the heap \passthrough{\lstinline!let b = Box::new(A \{ ... \});!}.
The binding that "holds the box" is the unique owner  of that data but because
boxes have move semantics, the ownership can be transferred across function
scopes.

\subsubsection{References}

Reading and even writing data one does not own is still possible in Rust but has
to be done with references. In other words, references are the way to --
temporarily -- break away from the ownership
invariant~\cite{lightweight-formalism}. References are created by
\emph{borrowing} from a variable or field with the \lstinline!\&! or
\lstinline!\&mut! operator, depending on whether the created borrow should be
immutable or mutable. One can either borrow from owned data directly or
\emph{reborrow} from an existing reference. When passing a reference to an
object as a function argument, we say we pass that argument \emph{by reference},
as opposed to passing the entire object \emph{by value} and hence moving or
copying it.

Rust's type system and borrow checking guarantee that all references are valid
when they are used. That is use-after free errors are impossible and Rust also
prevents stack-allocated data from escaping its allocation scope. To do so, Rust
needs to infer the lifetime of each reference. \emph{Liveness} is a general
concept in programming languages and can be determined by a \emph{data-flow
analysis}. A variable or for this case a reference is said to be \emph{live} if
it holds a value that may be used at any future point of any execution path
\cite{wiki:live-vars}. Otherwise, that variable is said to be \emph{dead}. The
part of the program in which a variable is live is called its \emph{lifetime}.

The lifetime of a reference in Rust is constrained by the lifetime of the
variable it borrows from, this is the main  mechanism to prevent use-after free
problems. Furthermore, there is a very strict set of rules that borrows need to
adhere to. At any point in a program and for any data there can either exist

\begin{itemize}
\tightlist
\item
  multiple immutable i.e.~\emph{shared references}
  \passthrough{\lstinline!\&T!} that may read the referenced data,
\item
  or there can be a \emph{unique mutable reference}
  \passthrough{\lstinline!\&mut T!} that is allowed to read and write
  to the data it does not own.
\end{itemize}

During the lifetime of a shared reference, the data can be read but not
changed by the owning binding. In contrast, the mutable reference allows
to change data without owning it under the condition that there exist no
other references to that data. While the mutable reference is live, even
the owning binding is neither allowed to read nor to write that place.

These rules underpin how Rust can ensure memory safety at compile time
and therefore form the core of the language. The type and borrow checker
of the Rust compiler enforce the rules or in the words of \citet{oxide}:

\begin{quote}
{[...]} we understand Rust's borrow checking system as ultimately being a system
for statically building a proof that data in memory is either
\emph{uniquely owned} (and thus able to allow unguarded mutation) or
\emph{collectively shared}, but not both.
\end{quote}

\subsection{Rust Compared to Scala}

\subsubsection{Mutability}

In Scala, there are local mutable variables (\passthrough{\lstinline!var!})
comparable to \passthrough{\lstinline!let mut!} in Rust. Class fields however
are immutable by default. At the class declaration one can opt into mutability
for a certain field by marking it as \passthrough{\lstinline!var!}. That field
is then mutable on all instances of the class and irrespective of whether an
instance is locally bound with the immutable \passthrough{\lstinline!val!} or
the mutable \passthrough{\lstinline!var!}.

\subsubsection{References}

Places or lvalues are not programatically accessible in Scala, as in JVM
languages there is no way to take a reference to a place explicitly. That means
it's impossible to generalise over whether one wants to assign to a local
variable or an object field by taking a reference to it.

As in all JVM languages, objects in Scala are heap-allocated and function
parameters are always passed by value. However, for objects this value is just a
reference to the object in the heap. This is very efficient for sharing
immutable objects while retaining by-value semantics. But as a consequence, the
by-value passing of references closely resembles by-reference passing for
mutable objects. The translation presented in \autoref{translation} will
extensively exploit this fact.

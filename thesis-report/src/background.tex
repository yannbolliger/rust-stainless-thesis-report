This chapter offers a short introduction to the Rust language as well as the
Stainless verifier. Both parts will use a running example of a Rust program
(\autoref{running-example}) to illustrate the involved concepts. It may seem odd
to introduce Stainless with Rust  as it is built and usually used with Scala.
However, thanks to Rust-Stainless the  usage of Stainless in Rust is so similar
to the usage in Scala that it is suitable  to reuse the running Rust example.
For a more detailed introduction to Stainless in Scala please refer to its
documentation \cite{stainless-doc}.

\section{Rust Language}

Rust \cite{rust2, rust1} is a recent systems programming language, initially
developed at Mozilla with the goal of replacing C\texttt{++} as primary
application language. Its most distinguishing feature is the new approach to
memory management. Traditional languages tend to either manage memory with a
\emph{garbage collector} or require the programmer to manually allocate and,
more importantly, deallocate memory. Rust deallocates memory itself without a
garbage collector by using concepts of \emph{ownership} and \emph{reference
lifetimes}. In other words, Rust provides memory management at compile-time
which results in cost-free abstractions at runtime. Additionally, Rust
guarantees memory safety properties like  the absence of \emph{dangling
pointers}, i.e. it is impossible to dereference a freed part of memory.

To live up to its promises, Rust introduces a flow-sensitive type checker, the
\emph{borrow checker}, which is a concrete implementation of ideas that have
been studied in research for years. The two main concepts are ownership
\cite{ownership-types} and \emph{borrowing} that build on the research topics of
\emph{uniqueness} \cite{alias-burying} and \emph{linearity}
\cite{Wadler90lineartypes, once-upon-a-type}. More details on the research
background are in \autoref{related-work}.


\subsection{Syntax Overview}

\autoref{lst:first-ex} shows a simple Rust function that returns the square of
the integer \lstinline!x!. Rust is statically typed, hence parameter types are
annotated: \lstinline!i32! in this case, a signed 32-bit integer. Other integer
types can be unsigned (\lstinline!u32!) or of other bit-lengths
(\lstinline!u128!). The return type of the function is indicated after the
arrow, if omitted it defaults to \lstinline!()!, the unit type. Note also, that
the single (and last) expression in the function block is implicitly returned.
Alternatively, one may employ the explicit return statement to return a value
from any point in a function, like on line 4 of \autoref{lst:let}. Statements
are delimited with semi-colons and produce the unit type.

\begin{lstlisting}[style=short, language=Rust, caption={A simple Rust function.}, label=lst:first-ex]
fn square(x: i32) -> i32 { x * x }
\end{lstlisting}

\subsubsection{Variables}

The \lstinline!let! statement is used to create a new variable and bind the
assigned value to it. Function parameters are also variables in
Rust~\cite{rustref}. By default, variables are immutable, that is, the compiler
rejects any attempt to reassign to an initialised variable. On the other hand,
Rust accepts to redeclare and thereby \emph{shadow} a binding, which is what
happens on line 7 of \autoref{lst:let}.

The compiler infers the type of most let-bindings, such that the type does not
have to be annotated like for function parameters. In \autoref{lst:let}, all the
types are inferred to be the standard integer type, \lstinline!i32!. This even
works for bindings that shadow an earlier declaration of a different type.

\noindent\begin{minipage}[t]{.45\textwidth}
\begin{lstlisting}[language=Rust, caption={A Rust function doing some arithmetics.}, label=lst:let]
fn f(z: i32) -> i32 {
  let x = z * 10;
  if x > 100 {
    return z;
  }
  let y = z * 100;
  let z = 3;
  x + y + z
}
\end{lstlisting}
\end{minipage}\hfill
\begin{minipage}[t]{.45\textwidth}
\begin{lstlisting}[language=Rust, caption={Mutable variable bindings.}, label=lst:mut-ex]
fn g(mut a: i32) -> i32 {
  let mut b = 10;
  a += b;
  b = 100;
  a * b
}
\end{lstlisting}
\end{minipage}

\subsubsection{Mutability}

As an imperative language, Rust also offers mutable bindings. Mutable variables
have to be explicitly declared with the \lstinline!mut! keyword. In general,
Rust is very explicit about the distinction between mutable and immutable
variables (and references). \autoref{lst:mut-ex} shows both a mutable function
parameter and a mutable local variable.

\subsubsection{Algebraic Data Types (ADTs)}

Taking strong inspiration from C, user-defined data types in Rust are called
\lstinline!struct!s and \lstinline!enum!s. Structs are product types and come in
three forms: without fields, like \lstinline!OnlyAMarker! in
\autoref{lst:struct-ex}, so called \emph{unit structs}; the tuple form with
numerically indexed fields, e.g. \lstinline!a.0!; and the C-like structs with
field names, \lstinline!b.field!. Enumerations are sum types and have members
that are again in either unit, tuple or C-like form. \autoref{lst:option} shows
an enumeration example from the standard library, the option type. It also shows
how the type parameter \lstinline!T! is used to create a generic data type.

\noindent\begin{minipage}[t]{.45\textwidth}
\begin{lstlisting}[language=Rust, caption={All three forms of structs.}, label=lst:struct-ex]
struct OnlyAMarker;
struct A(i32, bool);
struct B {
  field: u8
}
let a = A(0, true);
let mut b = B { field: 2 };
\end{lstlisting}
\end{minipage}\hfill
\begin{minipage}[t]{.45\textwidth}
\begin{lstlisting}[
  language=Rust,
  caption={The standard option type as an example of a generic enumeration.},
  label=lst:option
]
enum Option<T> {
    None,
    Some(T),
}
let opt: Option<i32> =
  Option::Some(123);
\end{lstlisting}
\end{minipage}

Fields are not considered variables in Rust, rather they are a part of their
corresponding struct's variable~\cite{rustref}. Hence, fields inherit properties
like mutability from their struct, e.g. the only mutable field in
\autoref{lst:struct-ex} is \lstinline!b.field! because \lstinline!b! is mutable.
Thus, mutability is not part of the field type but depends on the binding of the
corresponding struct's instance.

Rust also has anonymous tuples. These behave exactly like tuple structs but
don't need to be declared beforehand. One can simply instantiate tuples of any
non-negative number of types (cf. \autoref{lst:tuples}). This is used in the
first part of the running example (\autoref{lst:container}) that is a generic
container struct with an option of a tuple as field.

\noindent\begin{minipage}[t]{.45\textwidth}
\begin{lstlisting}[language=Rust, style=short, caption={A 2-tuple in Rust.}, label=lst:tuples]
let t: (i32, bool) = (123, true);
\end{lstlisting}
\end{minipage}\hfill
\begin{minipage}[t]{.45\textwidth}
\begin{lstlisting}[language=Rust, style=short, label={lst:container}, caption={The struct for the running example.}]
struct Container<K, V> {
  pair: Option<(K, V)>,
}
\end{lstlisting}
\end{minipage}

\subsubsection{Pattern Matching}

With ADTs, especially enumerations, pattern matching comes naturally.
\autoref{lst:match} shows how to match on the container struct from before. Like
in Scala, the underscore is the wildcard pattern and match arms can be  refined
by \lstinline!if! guards of any boolean expression. To merely check whether a
value matches a certain pattern, one can use the
\passthrough{\lstinline"matches!(opt, Some(_))"} macro that is expanded to a
regular pattern match which returns true if the given pattern matches and false
otherwise.

\begin{lstlisting}[language=Rust, caption={Pattern matching on a struct.}, label=lst:match]
let c = Container { pair: None };
match c {
  Container { pair: Some((k, v)) } if k == 123 => Some(v),
  _ => None,
}
\end{lstlisting}

\subsubsection{Implementations and Traits}

Implementation blocks add methods to data types in Rust. There can be many such
blocks for a given type. \autoref{lst:impl1} adds three methods to the container
struct of the running example. The implementation block is generic in two type
parameters \passthrough{\lstinline!K, V!}, as is the struct. Note that adding
methods only for certain instantiations of the  type parameters is also
possible. For example, the first line of the block could be
\passthrough{\lstinline!impl Container<bool, i32>!}, to only add methods to
containers of boolean keys and integer values.

The \lstinline!Self! in the first return type stands for the type this block is
adding methods to, in the example that is \lstinline!Container<K, V>!. The
\lstinline!new! method is a static method, it is not called on instances of
containers. In contrast, the two other methods specify a receiver type as first
parameter. That makes the methods available on instances of the container, for
example \lstinline!c.is_empty()!. In methods with the \lstinline!&self!
receiver, the \lstinline!self! keyword contains a reference to the instance on
which the method was called, in the example of \lstinline!c.is_empty()!, this is
equivalent to \lstinline!is_empty(&c)!. The same happens for
\passthrough{\lstinline!&mut self!} but mutably.\footnote{More on references in
\autoref{references}.} Finally, it is also possible to consume the instance by
specifying \lstinline!self! as receiver.

\begin{lstlisting}[language=Rust, caption={Methods for the running example.}, label=lst:impl1]
impl<K, V> Container<K, V> {
  pub fn new() -> Self { Container { pair: None } }
  pub fn is_empty(&self) -> bool { matches!(self.pair, None) }
  pub fn insert(&mut self, k: K, v: V) { self.pair = Some((k, v)) }
}
\end{lstlisting}

Traits are used to specify interfaces in Rust. A trait \lstinline!X! can be
implemented for a data type \lstinline!A! with an \passthrough{\lstinline!impl X
for A!} block, providing implementations for the abstract methods of the trait,
as in \autoref{lst:trait-impl2}. Traits can also be used to bound type
parameters to types that implement a certain trait. For example, the trait bound
\lstinline!K: Id!  in \autoref{lst:mut-ref-impl} states that any \lstinline!K!
admissible to the function (or in other cases the block) needs to implement the
\lstinline!Id! trait, i.e.~\passthrough{\lstinline!impl Id for K!} needs to be
available.

\noindent\begin{minipage}[t]{.45\textwidth}
\begin{lstlisting}[language=Rust, style=short, caption={Rust trait with one abstract method.}, label=lst:trait-impl1]
trait Id {
  fn id(&self) -> usize;
}
\end{lstlisting}
\end{minipage}\hfill
\begin{minipage}[t]{.45\textwidth}
\begin{lstlisting}[
  language=Rust,
  style=short,
  caption={Non-sensical implementation of the \texttt{Id} trait for strings.},
  label=lst:trait-impl2
]
impl Id for String {
  fn id(&self) -> usize { 123456 }
}
\end{lstlisting}
\end{minipage}

\subsection{Owning and Referencing Data}

\subsubsection{Value Categories}

Rust, like C but unlike Scala or any language on the \emph{Java Virtual Machine
(JVM)}, exposes memory management to the programmer and allows referencing
stack-allocated data. Therefore, one needs to distinguish between the different
\emph{value categories} of an expression. The general definitions stemming from
C are:

\begin{itemize}
\tightlist
\item
  \emph{lvalues} are expressions that designate memory locations, for
  example variables, array elements etc. They are objects that have a
  storage address. The term originates from the fact that lvalues are on
  the left-hand side of an assignment.
\item
  \emph{rvalues} are expressions or temporary values that do not
  persist. They are on the right-hand side of an assignment \cite{wiki:lvalues}.
\end{itemize}

These two terms were replaced in Rust by \emph{place and value expressions}.
Value expressions represent actual values and are defined by exclusion from
place expressions. On the other hand, place expressions evaluate to a
\emph{place} which is essentially a memory location~\cite{rustref}.

\begin{quote}
{[}Place{]} expressions are paths which refer to local variables,
{[}\ldots{]} dereferences (\passthrough{\lstinline!*expr!}), array
indexing expressions (\passthrough{\lstinline!expr[expr]!}), field
references (\passthrough{\lstinline!expr.f!}) {[}\ldots{]}. \\
\cite[section ``Expressions'']{rustref}
\end{quote}

Place expressions occur in \emph{place expression contexts} like on the
left-hand side of let-bindings, as borrow operands or as field expression
operands. Otherwise, if a place expression is evaluated in a \emph{value
context}, e.g.~on the right-hand side of a binding, it evaluates to what is
stored at the place it designates. In other words, its data is \emph{used}.

\subsubsection{Ownership}

As said previously, Rust's distinguishing feature is its ownership system which
is responsible for

\begin{quote}
{[enforcing]} an ownership invariant where a
variable is said to ``own'' the value it contains such that no two variables can
own the same value \cite[page 5]{lightweight-formalism}.
\end{quote}

In other words, data is owned by exactly one binding and if some data is used,
e.g. in an assignment or passed to a function, it will be \emph{moved out} of
that binding, i.e. the ownership of the data is transferred to the new binding.
The old binding is de-initialised and can never be reused. Moveable data behaves
linearly \cite{oxide}, moving out can be seen as \emph{destructive read}
\cite{islands-alias-protection}. On line 4 of \autoref{lst:move}, data is moved
out of \lstinline!a! and its ownership is transferred to \lstinline!b!.

From the example it is clear, why Rust's borrow checker needs to be
flow-sensitive. The move semantics render \lstinline!a! dead before it formally
goes out of scope, e.g. at the end of the block. The same applies to individual
fields. A field can be moved out of its struct and become inaccessible while
another field of the same struct is still available. This is called a
\emph{partial move}. The struct itself also becomes inaccessible as soon as one
of its fields is moved.

\noindent\begin{minipage}[t]{.47\textwidth}
\begin{lstlisting}[
  language=Rust,
  label=lst:move,
  caption={A struct type with move semantics.}
]
let a = Container {
  pair: Some(123, 456)
};
let b = a;
// `a` can never be used again
\end{lstlisting}
\end{minipage}\hfill
\begin{minipage}[t]{.47\textwidth}
\begin{lstlisting}[
  language=Rust,
  label=lst:copy,
  caption={Copy semantics of the \lstinline!i32! type.}
]
let a = 123;
let mut b = a;
b += 1;
// `a` can still be used
assert!(a == 123 && b == 124)
\end{lstlisting}
\end{minipage}

\paragraph{Move vs Copy}

Moveable types have \emph{move semantics}. One departure from that are copyable
types. Types that implement the \lstinline!std::marker::Copy! trait get
\emph{copy semantics}. The compiler automatically implements that trait for
primitive types\footnote{Primitive types in Rust are booleans, characters,
\lstinline!str!, numeric types and the panic type \texttt{!}~\cite{rustref}.}
(cf. \autoref{lst:copy}) but also for shared references and tuples of copyable
types \cite[section "Special types and traits"]{rustref}. When copyable data is
used, it's implicitly copied and the original binding stays valid and
independent. In contrast to shared references that are copyable, mutable
references have move semantics because they need to stay unique, as will become
clear shortly. Non-copyable types can still be explicitly copied by implementing
the \lstinline!std::clone::Clone! trait and calling the \lstinline!clone!
method.

\paragraph{Heap Allocation}

Up until now, all data was stack-allocated and hence, tied to the scope of the
surrounding function. To outlive a function, data needs to be heap-allocated
which is achieved in Rust with the \lstinline!Box<T>! type. Everything that is
put into a box is directly allocated on the heap (cf. \autoref{lst:box}). The
binding that ``holds the box'', \lstinline!c_heap! in the example, is the unique
owner of that data. Because boxes have move semantics, the ownership can be
transferred to other bindings, e.g. across function scopes.

\begin{lstlisting}[language=Rust, label=lst:box, caption={A heap-allocated container.}, style=short]
let c_heap = Box::new(Container { pair: None });
\end{lstlisting}

\subsubsection{References}
\label{references}

It is possible to read and even write data one does not own in Rust through
references. In other words, references are the way to -- temporarily -- break
away from the ownership invariant~\cite{lightweight-formalism}. References are
created by \emph{borrowing} from a variable or field with the \lstinline!&! or
\lstinline!&mut! operator, depending on the desired mutability of the created
reference. One can either borrow from owned data directly or \emph{reborrow}
from an existing reference. Passing a reference to a function as an argument is
called pass \emph{by reference}, as opposed to passing the entire designated
object \emph{by value} and thereby moving or copying it. The method of
\autoref{lst:mut-ref-impl} receives a mutable reference to the container
instance in \lstinline!self! and returns a reborrowed mutable reference to the
second part of its tuple.

\begin{lstlisting}[
  float,
  language=Rust,
  label={lst:mut-ref-impl},
  caption={An implementation block with a trait bound. The method returns a mutable reference to some interior part of the struct.}
]
impl<K: Id, V> Container<K, V> {
  pub fn get_mut_by_id(&mut self, id: usize) -> Option<&mut V> {
    match &mut self.pair {
      Some((k, v)) if k.id() == id => Some(v),
      _ => None,
    }
  }
}
\end{lstlisting}

Rust's type and borrow checking system guarantees that all references are valid
when they are used. That is, use after free errors are impossible and
stack-allocated data cannot escape its allocation scope. To achieve this, Rust
infers the \emph{lifetime} of each reference, i.e. the part of the program in
which the reference may still be used at a future point of any execution
path~\cite{wiki:live-vars}. A reference or variable that is not used anymore is
said to be \emph{dead}. The lifetime of a reference in Rust is constrained by
the lifetime of the variable it borrows from. This is the main  mechanism to
prevent use after free problems. Furthermore, references need to adhere to a
strict set of rules. At any point of a program and for any data there can either
exist:


\begin{itemize}
\tightlist
\item multiple immutable, i.e.~\emph{shared references}, \lstinline!&T!,  that
may read the referenced data,

\item or a \emph{unique mutable reference}, \passthrough{\lstinline!&mut T!},
that is allowed to read and write to the data it does not own.
\end{itemize}

During the lifetime of a shared reference, the borrowed data can be read but not
changed by the owning binding. In contrast, the mutable reference allows to
change data without owning it, under the condition that no other references to
that data exist. While the mutable reference is live, even the owning binding is
neither allowed to read nor to write to that place. In the words
of~\citet{oxide}:

\begin{quote}
{[...]} we understand Rust's borrow checking system as ultimately being a system
for statically building a proof that data in memory is either
\emph{uniquely owned} (and thus able to allow unguarded mutation) or
\emph{collectively shared}, but not both.
\end{quote}

\paragraph{In-Place Updates}

It is common to implement mutable data structures in Rust with methods that
alter the structure via a \passthrough{\lstinline!&mut self!} reference, for
example \lstinline!insert! in \autoref{running-example}. Such in-place updates
are simple, if the method just needs to change some struct field, like in the
example.

However, the method often needs to temporarily consume some part of the
\lstinline!self!, in order to change it. Imagine a method on \lstinline!Option!
that swaps a new value into a \lstinline!Some! and returns the old option value.
The naive approach to that method in \autoref{lst:swap-naive} does not compile
because the ownership system does not allow moving out of a mutable reference,
which happens on line 3. This makes sense, because the mutable reference needs
to stay valid, hence the data cannot disappear by moving out.

\begin{figure}[hbt]
\noindent\begin{minipage}[t]{.48\textwidth}
\begin{lstlisting}[
  language=Rust,
  label=lst:swap-naive,
  caption={Naive swap implementation that does not compile.}
]
impl<T> Option<T> {
  fn swap(&mut self, t: T) -> Self {
    let tmp = *self;
    *self = Option::Some(t);
    tmp
  }
}
\end{lstlisting}
\end{minipage}\hfill
\begin{minipage}[t]{.48\textwidth}
\begin{lstlisting}[
  language=Rust,
  label=lst:swap-replace,
  caption={Swap using the replace method.}
]
impl<T> Option<T> {
  fn swap(&mut self, t: T) -> Self {
    std::mem::replace(
      self,
      Option::Some(t)
    )
  }
}
\end{lstlisting}
\end{minipage}
\end{figure}

The solution is to use the special purpose function
\lstinline!std::mem::replace! that atomically replaces the value in a mutable
reference with its second argument and returns the old value of the mutably
borrowed place. For the \lstinline!swap! in \autoref{lst:swap-replace}, this is
already the entire method.


\subsection{Rust Compared to Scala}

\subsubsection{Mutability}

Scala has local mutable variables (\lstinline!var!) comparable to
\passthrough{\lstinline!let mut!} in Rust. Class fields however are immutable by
default. One can opt into mutability for a certain field by marking it as
\lstinline!var! at the class declaration. Such a field is then mutable on all
instances of the class and irrespective of whether its corresponding instance is
locally bound with the immutable \lstinline!val! or the mutable \lstinline!var!.

\subsubsection{References}

Places or lvalues are not programatically accessible in Scala, as in JVM
languages there is no way to take a reference to a place explicitly. That means
it's possible a priori to generalise over whether one wants to assign to a local
variable or an object field by taking a reference to that lvalue/place.

As in all JVM languages, objects in Scala are heap-allocated and function
parameters are always passed ``by value''. However, for objects this value is
just a reference to the object in the heap. This is very efficient for sharing
immutable objects while retaining by-value semantics. But as a consequence, the
by-value passing of references closely resembles by-reference passing for
objects, which is mainly important for mutable objects. The translation
presented in \autoref{translation} will extensively exploit this fact.





\section{Stainless Verifier}

With a solid understanding of Rust, it is time to introduce verification.
Stainless \cite{stainless} is a formal verification tool for Scala. It lets
programmers add contracts or specifications to functions and data structures.
Stainless transforms the input program in multiple phases to a purely functional
language understood by its backend \emph{Inox} \cite{inox} which then tries to
prove or disprove that the contracts hold by using Z3 \cite{z3} or a similar
solver.

\subsection{Specifications}

The simplest example of a specification is shown in \autoref{lst:fact-scala}.
The expression in \lstinline!require! is called a \emph{precondition}, it has to
hold when the function is called; the \lstinline!ensuring! is a
\emph{postcondition} that must hold after the function if the precondition
holds. Simply put, Stainless establishes just the latter: that the
postconditions hold assuming the preconditions hold. Of course, Stainless is
more sophisticated and additionally proves that the program finishes for all
inputs (\emph{termination}), that preconditions hold at static call sites of
functions, that pattern matches are exhaustive, and that the program does not
crash at runtime (except for out-of-memory errors) \cite{stainless-doc}. If
these properties are invalid, then Stainless finds a counter-example input for
the crashing or faulty function.

\noindent\begin{minipage}[t]{.47\textwidth}
\begin{lstlisting}[
  language=Scala,
  label=lst:fact-scala,
  caption={The factorial with specifications in Scala.}
]
def fact(x: Int): Int = {
  require(x >= 0 && x < 10)
  if (x <= 0) 1
  else fact(x - 1) * x
} ensuring { r => r >= 0 }
\end{lstlisting}
\end{minipage}\hfill
\begin{minipage}[t]{.47\textwidth}
\begin{lstlisting}[
  language=Rust,
  label=lst:fact-rust,
  caption={The same factorial in Rust.}
]
#[pre(x >= 0 && x < 10)]
#[post(ret >= 0)]
fn fact(x: i32) -> i32 {
  if x <= 0 { 1 }
  else { fact(x - 1) * x }
}
\end{lstlisting}
\end{minipage}

As \autoref{lst:fact-rust} illustrates, the same function and specification is
also accepted by Rust-Stainless, which subsequently translates the program to
Stainless and lets it prove its correctness. The only difference is that
specifications are added as function attributes rather than inline statements
and that postconditions in Rust automatically have a \lstinline!ret! variable in
scope that represents the return value. Otherwise, the two are very similar
which, for simplicity, allows to introduce Stainless in the following sections
only with Rust code examples. In both languages, one can put assertions into
function bodies in addition to specifications, in Rust with the
\lstinline"assert!" macro. Stainless also proves that body assertions hold.

Up until now, this section has been imprecise about Stainless's input language.
Stainless does not support the full Scala language but rather a functional
subset called \emph{PureScala}. The expressions in specifications need to be in
PureScala. However, Stainless was extended to support some form of mutability
\cite{regb} which this project relies on. Chapter \ref{translation} will go into
details about mutability.

The methods of the container in the running example (fully displayed in
\autoref{running-example}) show how the features from above are applied to add
specifications. The \lstinline!implies! function in line 26 is a library helper
method provided by the \lstinline!stainless! crate. Body assertions are used in
the main function to ensure that the code behaves as expected.

Another library helper is the \lstinline!old! function. This is very useful in
postconditions to refer to the value of a mutable reference before the function
was executed. For example in \autoref{lst:old}, the \lstinline!old(i)! will
return the value of \lstinline!i! before the function and  \lstinline!i! will
have the value after the function.

\begin{lstlisting}[
  language=Rust,
  caption={The \lstinline!old! helper for postconditions with mutable parameters.},
  label=lst:old,
]
#[post(*i == *old(i) + 1)]
fn change_int(i: &mut i32) {
  *i = *i + 1;
}
\end{lstlisting}

\subsection{Algebraic Properties}
\label{laws-intro}

Traits are the way to specify mandatory interfaces in both Scala and Rust.
Common properties like equality and ordering are often modelled with traits.
While the languages can enforce implementors of traits to provide all abstract
methods, they cannot enforce higher-level contracts. For example, for an
equality trait with \texttt{eq} one would assume that all implementations are
reflexive, i.e. $\forall x: x~\mathtt{eq}~x$. However, both compilers cannot
guarantee that.

\begin{lstlisting}[
  float=bt,
  language=Rust,
  caption={Example trait with laws.},
  label={lst:liskov1}
]
trait Rectangle {
  #[post(ret > 0)]
  fn width(&self) -> u32;
  #[post(ret > 0)]
  fn height(&self) -> u32;

  fn set_width(&self, width: u32) -> Self;
  fn set_height(&self, height: u32) -> Self;
  #[law]
  fn preserve_height(&self, any: u32) -> bool {
    self.set_width(any).height() == self.height()
  }
  #[law]
  fn preserve_width(&self, any: u32) -> bool {
    self.set_height(any).width() == self.width()
  }
}
\end{lstlisting}


As a solution, one can specify contracts on traits in Stainless. First, trait
methods can have specification attributes like regular functions. These will be
proven to hold for all implementations. Furthermore, one can attach \emph{laws}
to traits, i.e.~algebraic properties that implementors need to
fulfil~\cite[section "Specifying Algebraic Properties"]{stainless-doc}.
Stainless also proves the correctness of the laws for each implementation of the
trait.

The last addition to the running example, is a law on the \lstinline!Id! trait
stating  that ids need to be positive. Of course, the toy implementation that
hard-codes an id for strings, holds that contract (lines 7-8 in
\autoref{running-example}). A more interesting example of laws in
\autoref{lst:liskov1} shows a violation of the  Liskov Substitution principle
\cite{liskov}. The trait defines four methods. Two of them have regular
postconditions and the laws state that a change to the width should not change
the height and vice versa. Clearly, the implementation in \autoref{lst:liskov2}
conforms to these properties. Without the laws, it would be tempting to add a
square implementation, as in \autoref{lst:liskov3}. However, the square changes
both dimensions at once -- violating the laws. Stainless detects this, the
square implementation fails verification, and thereby a possible bug is caught.
\newpage

\begin{figure}[hbt]
\noindent \begin{minipage}[t]{.49\textwidth}
\begin{lstlisting}[
  language=Rust,
  caption={Example implementation of the rectangle trait.},
  label={lst:liskov2},
  basicstyle=\footnotesize\ttfamily,
]
struct Rect {
  width: u32,
  height: u32
}
impl Rectangle for Rect {
  fn width(&self) -> u32 { self.width }
  fn height(&self) -> u32 { self.height }
  fn set_width(&self, width: u32) -> Self {
    Rect { width, height: self.height }
  }
  fn set_height(&self, height: u32) -> Self {
    Rect { width: self.width, height }
  }
}
\end{lstlisting}
\end{minipage}\hfill
\begin{minipage}[t]{.49\textwidth}
\begin{lstlisting}[
  language=Rust,
  caption={Example implementation violating the laws.},
  label={lst:liskov3},
  basicstyle=\footnotesize\ttfamily,
]
struct Square {
  width: u32
}

impl Rectangle for Square {
  fn width(&self) -> u32 { self.width }
  fn height(&self) -> u32 { self.width }
  fn set_width(&self, width: u32) -> Self {
    Square { width }
  }
  fn set_height(&self, height: u32) -> Self {
    Square { width: height }
  }
}
\end{lstlisting}
\end{minipage}
\end{figure}

\hfill \break \noindent This concludes the introduction to Rust and Stainless.
All Rust features needed for the motivating example (\autoref{running-example})
were introduced, as were the verification features of Rust-Stainless.  Most of
these features were only added in the course of this thesis project. The goal of
the remaining chapters is now to extend Rust-Stainless with all these features.
In particular, \autoref{translation} will explain how mutability and mutable
references are translated to Scala. Chapter \ref{implementation} distinguishes
between existing and added features, then gives some implementation details.

\lstinputlisting[
  float,
  language=Rust,
  caption={The full running example for \autoref{background} showcasing most of the newly introduced features of Rust-Stainless.},
  label={running-example}
]{code/thesis_example.rs}
